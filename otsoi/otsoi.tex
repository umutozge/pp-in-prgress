\documentclass[11pt,a4paper]{article}
%\documentclass[glov3,smallextended,nospthms,natbib]{svjour3}

\usepackage{stmaryrd}
\usepackage[T1]{fontenc}
\usepackage{umut,umuttr}
\usepackage{uling}
\usepackage{usynsem}
\usepackage{udrt}
\usepackage{utheorem}
\usepackage{mathptmx}
\usepackage{hyperref}
	\usepackage{xcolor}
\hypersetup{colorlinks=true,linkbordercolor=red,citecolor=magenta,linkcolor=cyan,pdfborderstyle={/S/U/W 1}}
\usepackage{fancyvrb}
\usepackage{comment}

\usepackage{tikz-qtree}
\usepackage[normalem]{ulem}

%linguex adjustment
\setlength{\Exlabelsep}{0.5em}

% for the paper problem
%\setlength{\hoffset}{50pt}
%\setlength{\voffset}{40pt}


\newcommand{\encspec}{Enç-specific}


\title{On the ``strength'' of indefinites: A view from Turkish}

\author{Umut \"Ozge}

% \institute{Umut \"Ozge
% 			\at Middle East Technical University \& University of Cologne
%  			\\ \email{tumuum@gmail.com}}
% 
% \titlerunning{``Strength'' of indefinites}
% \journalname{}

\date{\today -- PLEASE DO NOT CITE}


\begin{document}
\maketitle

\begin{abstract}
The paper claims that the \acc-marker in Turkish is an indicator of definiteness
minus uniqueness.
\end{abstract}

\section{Introduction}

Start with some general remarks on indefiniteness; then introduce the notion of
``strong'' indefiniteness; and then go on with Turkish, by saying that Turkish
has some interesting data that is relevant to this debate.

\exg. {\label{para-bare}John} {\bf kitap} {okudu.} \\
	{J.} {\bf book} {read}\\
`John did book-reading.'

\exg. {\label{para-def}John} {\bf kitab-{\i}}  {okudu.} \\
	{J.} {\bf book-Acc}	{read}\\
`John read the book.'

\exg. {\label{para-indef}John} {\bf bir} {\bf kitap} {okudu.}\\
	{J.} {\bf a}  {\bf book} {read}\\
`John read a book.'

\exg.{\label{para-acc}John} {\bf bir} {\bf kitab-{\i}} {okudu.}\\
	{J.} {\bf a} {\bf book-Acc} {read} \\
`John read a book.' (``strong'')

In this paper we are interested in \xref{para-acc}. Two facts make this form
interesting. One is that the \acc-marker is strongly associated with
definiteness in Turkish. This is most apparent in the minimal pair
\xref{para-bare} \versus\ \xref{para-def}. Furthermore, for noun phrases that
are usually considered definite---in the sense of displaying definiteness
effect---the marker is obligatory. These constructions are:

\ex.
\a. proper nouns;
\b. pronouns and demonstratives;
\b. ``strong'' DPs (TODO: give a list).
\b. derived nominals;
\b. genitive possessive constructions.

The curious case here is genitive possessive constructions. They require the
marker but they can be indefinite. The others are definite. Therefore, with the
exception of the genitive possessive construction, the \acc-marker behaves as an
indicator of definiteness.

The second fact that makes \xref{para-acc} interesting is that the marker is
optional on indefinites for some verbs (\ref{para-indef} \versus\
\ref{para-acc}), exemplifying a case of Differential Object Marking
\cttxp{aissen03}. This optionality has certain interpretative effects (see
section~\ref{sec-review}).

The following research questions follow from the observations above:

\ex.\label{res-ques}
\a.\label{res-ques-syn} What governs the distribution of the \acc-marker? (When
is it required, when is it optional?)
\b.\label{res-ques-sem} What is the contribution of the marker in cases where it is optional?
\b.\label{res-ques-genposs} Why the marker is not optional for genitive possessive indefinites?


The aim of the present paper is to propose an analysis of the marker that
answers these three questions in a unified way.

\section{Description of the phenomena}

In this section I aim to provide a critical overview of the descriptive aspects of the \acc- versus \zero-marked indefinites, concentrating on their interpretative differences.


\subsection{Scope and word order}
\label{scscope}

As in many other languages, two types of indefinites differ in their scopal behavior. Slightly adapting from \cttx{ozge11}:

Let us take an intermediate scope example:

\exg. Çogu dilbilimci önemli bir problem(-i) çözen her makale-yi okudu.\\
most linguist important a problem(-Acc) solve.Rel every article-Acc read.3sg\\
`Most linguists read every article that solves an important problem.'

In the \acc-marked version: we have three readings: (i) a single problem; (ii) a possibly different problem per linguist; (ii) a possibly different problem per article. In the \zero-marked version: only the narrowest scope reading is available.



\subsection{Specificity} %chungladusaw still cite enc for specificity.
\label{scenc}

Let me start with an influential account of the interpretative effect
of \acc-marking in Turkish, \cttx{enc91}, which also draws some
important claims on noun phrase interpretation in general.
\ctnm{enc91} claims that a Turkish noun phrase carries the \acc-marker
if and only if it is ``specific''. \ctnm{enc91} equates her notion of
``specificity'' with ``partitivity'' and ``strong''/``weak'' (and also with
D(iscourse)-linking of \ctnm{pesetsky87}). Here is one of her examples
illustrating her notion of ``specificity'' (henceforth \encspec ity):

\exg.\label{exencintro}%
{Odam-a}  {birka\c{c}} {\c{c}ocuk} {girdi.}\\
{my-room-Dat} {several}  {child}  {entered}\\
`Several children entered my room.'

\ex.\label{exenc}
\ag.\label{exenca}{\.iki}  {k\i{}z-\textbf{\i}} {tan\i{}yordum}.\\
{two}  {girl-{\bf Acc}}  {knew.1sg}\\
`I knew two girls (among the children). '\hfill (\encspec)
\bg.\label{exencz}\#{\.Iki}  {k\i{}z}  {tan\i{}yordum.} \\
{two}  {girl}  {knew.1sg}\\
`I knew two girls.'\hfill (non-\encspec)


\ctnm{enc91} observes that in a discourse initiated by
\xref{exencintro}, only with the form \xref{exenca}, where the
indefinite \trtx{iki kız}{`two girls'} is \acc-marked, the girls are
understood to be coming from the set of children; otherwise, in the
absence of the marker as in \xref{exencz}, hearers tend to interpret
the girls to be out of the set of children introduced in
\xref{exencintro}. 

Although \ctnm{enc91} associates the maker with ``partitivity'',
\acc-marked indefinites are clearly different from ordinary
partitives. While an ordinary partitive like the one in
\xref{exordpart}  entails that the restrictor is larger than the
referents of the partitive, witness the translation, an
\acc-indefinite does not.

\ex.\label{exordpart} Kızlardan ikisini tanıyordum.

Observe the fact that the English translation for \xref{exenca} is not
`I knew two of the girls'; the sentence is non-committal on whether
there are more girls among the children than the two mentioned.
Therefore \acc-marker is not simply a partitivity indicating item.

Before I start a critical assessment of this important proposal, I
would like to cast it in a formal framework to clarify what we are
dealing with. \ctnm{enc91} extends the dynamic model of noun phrase
semantics, which associates every noun phrase with an index (or,
equivalently, a discourse referent) that gets bound by an operator
sourced outside of the NP semantics.  \ctnm{enc91} adds a second
index, standing for a \textbf{superset} of the first index. Let us
call the latter the superset index and the former referent index.
Here is the formal definition from \cttx{enc91}:

\ex.\label{encform} Every $[_{\text{NP}}\  \alpha ]_{\langle i,j\rangle}$ is interpreted as
$\alpha(x_i)$ and\\
$x_i \subseteq x_j$ if $\text{NP}_{\langle i,j\rangle}$ is plural;\\
$\{x_i\} \subseteq x_j$ if $\text{NP}_{\langle i,j\rangle}$ is
singular.

\ctnm{enc91} further claims that the usual definiteness feature ([+definite]
for familiar, [-definite] for novel) applies separately to both the referent
and the superset index. In this setting, a standard definite NP has
[+definite], and a standard indefinite has [-definite] on their both indices.
\encspec ity corresponds to the case where the referent index has [-definite]
and the superset index has [+definite].\footnote{\cttx{enc91} does not discuss
the case ([+definite], [-definite]). TODO:check.}

Let us see how the proposal works, over \xref{exenc}. The discourse opener
\xref{exencintro} contributes the simplified main DRS in \xref{exencmaindrs},
containing the referents for the speaker and a set of children:

\ex.\label{exencmaindrs}
\a. Several children entered my room.
\b. \udrs{s',z}{children'(z)}

First take the non-case-marked continuation:

\ex.\label{exenczerodrs}
\a. I knew two girls.\hfill (non-case-marked)
\b.
\udrs{s',z,x_1,x_2}{children'(z)\\
two\mbox{-}girls'(x_1) \quad x_1 \subseteq x_2 \quad know'(s',x_1)
}

Indices $x_1$ and $x_2$ are both indefinite carrying a novelty condition. We
assume that this novelty condition makes sure that the set type indices $x_1$
and $x_2$ are disjoint with any already established index, a condition not
included in (\ref{exenczerodrs}b).  Thanks to the novelty condition on both
indices, the girls are understood to be not included in the set of children
established in the discourse model, eventually giving rise to inappropriateness
in the context of \xref{exencintro}.

Let us now turn to the analysis of the case-marked continuation
\xref{exenca}.

\ex.\label{exencacc1}
\a. I knew two girls.\hfill (Acc-marked)
\b. \udrs{s',z,x_1,\anaph{x_2}}{children'(z)\quad  entered'(z)\\
two\mbox{-}girls'(x_1) \quad x_1 \subseteq \anaph{x_2} \quad know'(s',x_1)
}

Here the direct object \trtx{iki kız-ı}{two girl-Acc} again contributes two
indices, but this time the second index standing for a superset is definite. In
modelling definiteness of this sort, I follow the binding theory of
presupposition resolution of \cttx{sandt92}. I think at this stage nothing
crucial hinges on the choice of definiteness account and its formal
representation. In this theory, presupposition resolution is treated on a par
with anaphora resolution. Underlining a discourse referent, as is done in
\xref{exencacc1}, indicates that the underlined content needs an antecedent to
get bound. This binding relation between the antecedent and the
presuppositional content obeys the usual accessibility constraints of DRT.
Therefore, for the interpretation in \xref{exencacc1} to converge this binding
requirement needs to be satisfied. Presupposition resolution in this fashion is
a non-deterministic process that obeys certain restrictions(TODO:
citation):

\ex.
\a. The antecedent should be accessible.
\b. If there is an antecedent accessible in a near distance bind to that; unless there are inferential restrictions.
\b. If an antecedent is not present, accommodate one -- this accommodation process is guided again by inference.

In the present case the presupposition carried by the superset index $x_2$ is
resolved through identification with the set $z$ of children already
established in the model:

\ex.
\udrs{s',z,x_1,\anaph{x_2}}{children'(z)\\
two\mbox{-}girls'(x_1) \quad x_1 \subseteq \anaph{x_2} \quad know'(s',x_1)\\
\anaph{x_2} = z
}


If all the presuppositions of a DRS are bound, than the algebraic cancellation of identical terms yields a DRS without any underlined terms. This holds for the current case. The representation we arrive at correctly captures Enç's observation for the \acc-marked continuation to \xref{Exenc16}:

\ex.\udrs{s',z,x_1}{children'(z)\\
two\mbox{-}girls'(x_1) \quad x_1 \subseteq z \quad know'(s',x_1)
}

\ctnms{enc91} proposal has been empirically challenged on the basis of
\acc-marked out-of-the-blue indefinites as well as non-case-marked
indefinites that are yet \encspec\
(\cttx{taylanzimmer94,zidani97,kelepir01,heusingerkornfilt05,heusingerkornfilt17,kilicaslan06,issever07,nakipoglu09,ozge11}
among others).  In this paper, I will gloss over the gaps in the data
regarding the semantic effects of the \acc-marker and concentrate on
the cases where the presence of the marker has an effect related to
previous discourse that is absent for unmarked indefinites. I will
show that \encspec ity, as the semantic correlate of \acc-marking,
cannot hold up to the facts of Turkish under close scrutiny. I will
instead argue that the semantics of the marker is more closely related
to ``strong''/``weak'' distinction of \ctnm{milsark77}.  This might
appear paradoxical, given that \ctnm{enc91} equates her notion of
``specificity'' with ``strong''/``weak'' (and also with
D(iscourse)-linking of \ctnm{pesetsky87}), I will show, however, that
this equivalence does not hold.  \encspec ity\ is implicit domain
restriction, which is a weaker relation than ``strength''; and it is
not the semantic property that the Turkish \acc-marker indicates. In
order to proceed in this direction, first we need to look at \encspec
ity in more detail.

Partitivity is a side-effect rather than directly coded in the
semantics.

This is of course different than implicit domain restriction, as it
involves some familiarity for the way the domain is restricted.

Identifiability vs. familiarity.



\subsubsection{\encspec ity under negation}

Having seen how \encspec ity works for Enç's example \xref{Exenc17}, let us modify the example to be able to derive further predictions of her account. Here is a minimal pair, where the verb \trtx{tanı}{know} is negated:

\exg.\label{Exenc16neg}%
{Odam-a}  {birka\c{c}} {\c{c}ocuk} {girdi.}\\
{my-room-Dat} {several}  {child}  {entered}\\
`Several children entered my room.'

\ex.\label{Exenc17neg}
\ag.\label{Exencnegacc}{\.Iki}  {k\i{}z-\textbf{\i}} {tan\i{}-m-ıyordum}.\\
{two}  {girl-{\bf Acc}}  {knew.Neg.1sg}\\
`I didn't know two girls (among the children). '\hfill (case-marked)
\bg.\label{Exencnegzero}\#{\.Iki}  {k\i{}z}  {tan\i{}-m-ıyordum.} \\
{two}  {girl}  {knew.Neg.1sg}\\
`I didn't know two girls.'\hfill (non-case-marked)

Before commenting on the available interpretations of these negated variants of Enç's examples, let us first observe the interpretation assigned by her account. Let us assume that negation in \xref{Exenc16neg} is at VP level, which contains the verb and the direct object in both case-marked and non-case-marked variants. Under this assumption, the non-case marked \xxref{Exenc17neg}{b} gets the following interpretation:

\ex.\label{Exencnegzerodrs}
\udrs{s',z}{children'(z)\\
	\udrs[\neg]{x_1,x_2}{
	two\mbox{-}girls'(x_1) \quad x_1 \subseteq x_2  \quad know'(s',x_1)
	}
}

Given $x_2$ is [-definite] and therefore disjoint with $z$,
\xref{Exencnegzerodrs} is satisfied in a model where it is impossible to find
at least two girls not belonging to the children in the room such that the
speaker knows them. Therefore, \xref{Exencnegzerodrs} would be satisfied in a
model where there are girls among the children known to the speaker.  Then how
does this intepretation fair with what could actually be meant by
\xxref{Exenc17neg}{b}? If \xxref{Exenc17neg}{b} coheres with \xref{Exenc16neg}
at all, it means that there are not at least two girls among the children known
to him/her, with the proviso that this reading is available only under further
specification of the context such that whether the speaker knew two girls among
the children or not. In any event the interpretation delivered in
\xref{Exencnegzerodrs} is hardly satisfactory.

Let us now turn to the case-marked variant \xxref{Exenc17neg}{a}, which is
interpreted in three steps:

\ex.
\a.
\udrs{s',z}{children'(z)\\
	\udrs[\neg]{x_1,\anaph{x_2}}{
	two\mbox{-}girls'(x_1) \quad x_1 \subseteq \anaph{x_2}  \quad know'(s',x_1)
	}
}
\b.
\udrs{s',z}{children'(z)\\
	\udrs[\neg]{x_1,\anaph{x_2}}{
	two\mbox{-}girls'(x_1) \quad x_1 \subseteq \anaph{x_2}  \quad know'(s',x_1)\\
	\anaph{x_2} = z
	}
}
\b.\label{Exencnegaccdrs}
\udrs{s',z}{children'(z)\\
	\udrs[\neg]{x_1}{
	two\mbox{-}girls'(x_1) \quad x_1 \subseteq z  \quad know'(s',x_1)
	}
}

The DRS \xref{Exencnegaccdrs} is satisfied in any model where it is impossible
to find at least two girls known to the speaker and belong to the set of the
children at the same time. A critical observation here is that
\xref{Exencnegaccdrs} gets satisfied in a model where there were no girls at
all among the children. By this token, \xref{Exencnegaccdrs} diverges from the
actual meaning of \xref{Exencnegacc}: In its primary reading,
\xref{Exencnegacc} gets satisfied only if there are  at least two girls among
the children such that the speaker does not know them. In DRT notation:
\footnote{The reading \xref{Exencwide} is not the only reading the form
\xref{Exencnegacc} can get. I will discuss another type of reading for such
sentences below. What is crucial for now is that \xref{Exencnegacc} simply
lacks any reading that does \emph{not} commit to the existence of at least two
girls among the children.}


\ex.\label{Exencwide}
\udrs{s',z,x_1}{children'(z)\\
	two\mbox{-}girls'(x_1) \quad x_1 \subseteq z\\
	\udrs[\neg]{}{
	know'(s',x_1)
	}
}

Again, we end up with an unsatisfactory interpretation.

Is there a way to keep \encspec ity as it is and resort to other factors to
explain the incorrect predictions of the account? \ctnm{enc91} proposes such a
solution.\footnote{From this proposal, I infer that she is already aware of the
problem, although she does not discuss negation.TODO:check} One major aim of
\ctnm{enc91} is to motivate a notion of specificity that is orthogonal to scope
phenomena. It is well-known that the two notions are closely related
\cttx{farkas02a}). She first observes that \acc-marking indefinites tend to
take wide-scope. She explains this fact by claiming that in cases where
case-marking and non-marking yields the same interpretation, the case-marked
version is interpreted as taking wider-scope with respect to a commanding
operator, through a Gricean inference.

Adapting the argument to the present would yield,

\ex.\label{encgrice}
\a. \xref{Exencnegaccdrs} is indeed the semantic interpretation that would be assigned by the grammar to \xref{Exencnegacc};
\b. The non-case-marked version is assigned the same interpretation by the grammar. Therefore,  \xref{Exencnegaccdrs} and \xref{Exencnegzerodrs} are equivalent.
\b. Because of this equivalence, the hearer of \xref{Exencnegacc} unconsciously reasons as follows: ``The speaker could have expressed the same content with a non-marked version, but she uses the marked one. Therefore, she is trying to convey a non-standard interpretation, which I take to be the one where the indefinite takes scope over negation''. Through, this reasoning the hearer ends up with interpreting \xref{Exencnegacc} as \xref{Exencwide}.


An immediate problem with this argument concerns the equivalence of
\xref{Exencnegaccdrs} and \xref{Exencnegzerodrs}. This cannot be
truth-conditional equivalence, since in a model where the speaker knows two
girls outside of the children set, the former representation gets satisfied
while the latter does not. Therefore, the argument in \xref{encgrice} is at
least in need of specifying a notion of equivalence that would hold between
these two representations, such that it will serve the ground for the Gricean
reasoning proposed. Otherwise, the reason why \xref{Exencnegacc} is not
understood in the way predicted by \ctnms{enc91} account remains unexplained.

There is another source of the wide-scope behavior of the indefinite
in \xref{Exencnegacc}, namely the grammar. The indefinite could be
forced to move out of its local domain for case-checking or some other
reason along the lines of \ctnms{diesing92} Mapping Hypothesis or some
variant of it.\footnote{For the syntactic position of \acc-marked
indefinites see \cttx{kelepir01,aydemir04,ozturk05}.} Such an
independent motivation for the indefinite in \xref{Exencnegacc} to
take wide-scope would explain why \ctnms{enc91} prediction for the
example is not full-filled: The marker indicates \encspec ity, but at
the same time the indefinite is forced to move to a position higher
than the negation operator, and by this token \xref{Exencnegacc} gets
interpreted as \xref{Exencwide}, rather than as \xref{Exencnegaccdrs}.
At the moment, we do not need to get concerned about the exact
position the indefinite is forced to move. One possibility would be:

\ex.\begin{tikzpicture}
\tikzset{sibling distance=25pt}
\Tree [.{CP} [.C  {two girls} ]  
 			[.IP [.{NP} \textit{pro}  ] [.I$'$ [.{NegP}  	[.{VP} 
								[.{NP} {\sout{two girls}} ]
								[.V know ]
				  			] 
							{Neg}
					  ] 
					  [.I ]
				  ]
			]
	]
\end{tikzpicture}

\ex.\label{wscopeaccount}{\it Forced wide-scope account of \acc-marked indefinites:}
\a. The accusative case on Turkish indefinite direct objects marks \encspec ity, as formulated in \xref{encform}. 
\b. An accusative marked direct object is required to raise to a position that is at least higher than the verbal\footnote{Turkish has another negation operator \textit{değil} with sentential scope.} negation operator.


I will argue that \xref{wscopeaccount} cannot fully explain the semantic reflex
of the case-marker on the indefinites in Turkish. The reason is that
\acc-marked indefinites do not necessarily get a wide-scope reading as was the
case in \xref{Exencnegacc}. Such an argument requires a closer look at the
scopal semantics of \acc-marked indefinites, to which I will directly turn. Let
me diverge from \ctnms{enc91} example and  replace the numeral quantifier
\trtx{iki}{`two'} with the indefinite determiner \trtx{bir}{`a/one'} and
negation with antecedent of a conditional; I return to \ctnms{enc91} example
below.

\ex.Context: John is a referee reviewing an article under discussion.
\ag.\label{conda}John bir hata{\bf-yı} görür-se, makale-yi reddeder.\\
	J. a error{\bf-Acc} sees-Cond article-Acc rejects\\
	Rd.\ 1: `If John sees an error, he'll reject the article.'' (It's common
	ground that there are errors in the article.)\\
	Rd.\ 2: `An error is such that if John sees it, he'll reject the article.
\bg.\label{condz}John bir hata görür-se, makale-yi reddeder.\\
	J. a error sees-Cond article-Acc rejects\\
	`If John sees an error, he'll reject the article.'' (no commitment to the existence of errors.)


A crucial difference between \xref{conda} and \xref{condz} is that while the
former presupposes the existence of errors in the article, no such
presupposition is involved in the latter. Another important fact concerning
\xref{conda} is that besides a wide-scope reading of the indefinite, it has
another reading where the indefinite stays within the scope of the antecedent
of the conditional, while its restrictor stays out. The two readings are as
follows in DRT notation:

\ex. 
\a.\label{condawide}Wide scope reading:
\udrs{j',x,y}{article'(x) \quad error'(y)\\
\ccon
{\udrs{}{see'(j',y)}}
{\udrs{}{reject'(j',x)}}
{\Rightarrow}
}
\b.\label{condanar}Mixed scope reading:
\udrs{j',x,Y}{article'(x) \quad error'(Y)\\
\ccon
{\udrs{z}{z\in Y \quad see'(j',z)}}
{\udrs{}{reject'(j',x)}}
{\Rightarrow}
}


The non-case marked variant \xref{condz} is interpreted as follows:\footnote{An
interpretation along \cttx{enc91} would have an indefinite superset, which
would not have any truth-conditional effect.}

\ex. \udrs{j',x}{article'(x)\\
\ccon
{\udrs{y}{error'(y) \quad see'(j',y)}}
{\udrs{}{reject'(j',x)}}
{\Rightarrow}
}


The reading \xref{condanar}, which appears to be the primary reading for
\xref{conda}, is impossible to arrive at by coupling \encspec ity and
wide-scope behavior, namely \xref{wscopeaccount}. What is predicted by that
account instead is only \xref{condawide}.

\ctnm{enc91} actually has a similar example involving an intensional verb
\trtx{iste}{want}:

\ex.Context: A musical instrument store.
\ag. Ali bir piyano-yu kirala-mak istiyor.\\
A. a piano-Acc hire-INF wants\\
Rd.\ 1: `Ali wants to hire a piano from among those in the store (but he did not decide which).'\\
Rd.\ 2: `There is a piano from among those in the store such that Ali wants to hire it.'

Again, what I translate as ``Rd.\ 1'' is not possible to get via \xref{wscopeaccount}.

The scopal behavior of \acc-marked indefinites is quite general, it is
straightforward to have similar examples with other types of scope variation
inducing operators. Here is one with the imperative, assume a context of two
interviewers discussing what to ask to an interviewee:

\ex.
\ag.\label{impz}{On-a} {zor} {bir} {soru} {sor}.\\
{her-Dat} {hard} {a} {question} {ask}\\
`Ask her a hard question.'
\bg.\label{impa}{On-a} {zor} {bir} {soru\bf{-yu}} {sor}.\\
{her-Dat} {hard} {a} {question\bf{-Acc}} {ask}\\
`Ask her a hard question.'\hfill (D-linked)\\
Rd.\ 1: `Ask her one of the hard questions.'\\
Rd.\ 2: ??A specific question is intended, to be continued by naming the question
explicitly.


Let me return to negation. In order to avoid the interference of
meta-linguistic negation, I follow \ctnms{szabolcsi04} suggestion of using
reason contextualization. Take an examination context where the examinee looks
happy after the exam. When inquired about the source of her happiness, both
answers below are appropriate:  

\ex.
\a.\label{sorunegz}Öğretmen zor bir soru sor-ma-dı.\\
teacher hard a question ask-Neg-Past\\
`The teacher didn't ask a hard question.' (No commitment to the
existence of hard questions beforehand.)\\
\b.\label{sorunega}Öğretmen zor bir soru{\bf-yu} sor-ma-dı.\\
teacher hard a question{\bf-Acc} ask-Neg-Past\\
Rd.\ 1: `The teacher didn't ask a hard question.' (It's
common-ground that there were hard questions prepared beforehand that could be asked.)\\
Rd.\ 2: `There was a question such that (fortunately) the teacher didn't ask
it.'

Again the crucial observation is that \xref{sorunega} necessarily commits to
the existence of hard questions known to the examinee. No such existence
presupposition is present in the non-marked variant \xref{sorunegz}.

So far I have been reviewing the behavior of the \acc-marked indefinite in
relation to different types of operators, and demonstrating that there are
readings that are not accounted for by \xref{wscopeaccount}, namely mixed scope
readings where the restrictor is wide scope, but the indefinite is narrow
scope. In all the examples I have considered so far, the restrictor of the
indefinite and the contextually established set that the indefinite is linked
to are identical. In \ctnms{enc91} example \xref{Exen}, however, the restrictor
of the indefinite (\intx{girls}) and the antecedent set (\intx{children}) were
different. While discussing this example, I underlined the fact that the effect
of the marker was not simply a partitive reading like ``two of the girls.''
Here I provide a model example where this is again the case and we have a mixed
scope reading, as in other examples I discussed above. My aim is again to show
that \xref{wscopeaccount} cannot deliver all the available readings. 
 
Take a scenario where Alice has a number of dogs in her farm and her niece
Betty comes to visit her. Betty wants a dog among them as a birthday present
and she wants a Retriever. Alice gives the present, but Betty does not look
happy with what he get. Someone asks the reason for Betty's not being heppy.
Alice answers:\footnote{The verb \trtx{hediye et}{present make} is a light verb
construction that behaves identically to a lexical verb, as far as the grammar
of zero versus case-marked objects are concerned. Examples similar to this one
can be constructed with lexical verbs as well.}

\ex.
\ag.\label{retz}Çünkü on-a bir Retriever hediye et-me-dim.\\
 because her-Dat a Retriever present make-Neg-Past.1sg\\
`Because I didn't give her a Retriever as present.'
\bg.\label{reta}Çünkü on-a bir Retriever-ı hediye et-me-dim.\\ 
because her-Dat a Retriever-Acc present make-Neg-Past.1sg\\
`Because I didn't give her a Retriever as present.'


First, the \acc-marked version \xref{reta} is inappropriate in the absence of
the antecedent set of dogs; and it is ambiguous between a wide scope Retriever
reading and a mixed scope Retriever reading. Once again \xref{wscopeaccount}
can only deliver the wide-scope reading and therefore is not adequate in
capturing all the available readings.  

Finally, one might ask why a mixed-scope reading is not available for the
negation of \ctnms{enc91} example given as \xref{Exencnegacc} above.
Furrhermore, one might think that the absence of that sort of reading for
\xref{Exencnegacc} is due to having a numeral \trtx{iki}{two} instead of
\trtx{bir}{a/one}. The reason is that the typical context accommodated with the
examples cannot support that sort of reading. There is simply no potential
reason for the speaker to deny that she knows two girls from amont the children
entering the room. Once the needed contextual support is provided, mixed scope
readings arise for the numeral \trtx{iki}{two} as well.

For instance imagine a scenario where a bunch of children is to be assigned to
dormitory rooms that accommodates two. Also imagine that there is a rule
dictating not to put two girls in the same room, rooms should be either two
boys, a boy and a girl, or a single girl. The speaker of \xref{odaa} is
succesful in stating that she observed the rule applied to the bunch of
children already established in the discourse, given our scenario. In the
absence of such an established set of children, the approriate form would be
\xref{odaz} and the \acc-marked version \xref{odaa} would be
inappripriate.\footnote{I am grateful to Daniel Büring for pointing these types
of contexts out to me.} 

\ex.
\ag.\label{odaz}Aynı oda-ya iki kız koy-ma-dım.\\
same room-Dat  two girl put-Neg-Past.1sg\\
``I didn't put two girls in the same room.'
\bg.\label{odaa}Aynı oda-ya iki kız-ı koy-ma-dım.\\
same room-Dat  two girl-Acc put-Neg-Past.1sg\\
``I didn't put two girls in the same room.'


\subsection{Information structure}

% Conclude the section:
To summarize, we have seen that \acc-marked indefinites are flexible in taking scope, they trigger existence presuppositions, the interpretative effect of the marker is orthogonal to topicality.


\section{\acc-indefinites are ``strong''}
\label{scstrong}

As the examples we considered above show, \encspec ity misses the notion of
``existential import''.  In relation to an operator, \acc-indefinites receive
both a wide-scope reading and mixed scope reading, where the domain takes wide
scope while the discourse marker stays narrow-scope.  These facts immediately
suggest that the \acc-marked indefinites should be considered as
presuppositional or ``strong'' indefiniteness.  As we saw above,
\acc-indefinites project existence presuppositions at standard test contexts
like antecedent of a conditional, imperatives and negation. 

Within the light of these observations I claim that what is involved in
\acc-marked versus zero-marked indefinites is that the former type of
indefinites are ``strong''.  In an attempt to clarify the grammar of
existential sentences in English, \cite{milsark77} introduced the
``strong''/``weak'' distinction in determiners. ``Strong'' determiners are
those that quantify over a domain denoted by their restrictor terms. A typical
effect of ``strength'' is that the restrictor predicate is presupposed to exist
at the point of quantification. One can think of this as a two step process:
first you fix the domain, then you go on with quantification. ``Weak''
determiners on the other hand lack quantificational force of their own; what
they provide is a cardinality predicate that specifies the size of the
restrictor. ``Strongly'' determined NPs overlap with the syntactic/semantic
notion of definiteness but not completely. \ctnm{milsark77} credits to
\ctnm{postal66} the observation that ``strong''/``weak'' cross cuts the
territory of indefiniteness. 

In a DRT setting, the most straightforward way to model ``strength'' is to
adopt ``presuppositionality as anaphoricity'' perspective
(\cttx{sandt92,geurts99}).  According to this view, ``[a] strong quantifier
does not merely presuppose that its domain is non-empty; rather, the purpose of
its presupposition is to \emph{recover} a suitable domain from the context''
(\cttx[253]{geurts07}). Applying it to the present case yields:

\ex. The restrictor of an accusative indefinite is anaphoric.


It is straightforward to model this claim by a slight modification to
\ctnms{enc91} proposal. An \acc-marked indefinite is inserted to a DRS with a
presuppositional (=anaphoric) restrictor.  Going back to \ctnms{enc91} example,
the state of the discourse is as follows after the indefinite is inserted,

\ex.
\a. Several children entered my room.
\b. I knew two girls.\hfill (Acc-marked)

\ex.
\udrs{s',z,\anaph{x}, y}
{children'(z)\\
girls'(\anaph{x}) \quad two'(y) \quad y \subseteq \anaph{x} \quad know'(s',y)
}

Here the set of girls is presuppositional. As there is no established set of
girls in the context, one option is to accommodate one. I claim that it is by
inference that this accommodated set of girls is most naturally understood to
be included in the set of children introduced in the opening sentence of the
discourse. I do not have a systematic account of why the inference should yield
that result, apart from suggesting that it is the least costly assumption to
make to maintain the coherence of the text.  In formal terms, \xref{prencaccom}
depicts the accommodation of the antecedent set and the resolution of the
presupposition; \xref{prencfin} depicts the final form of the representation
after eliminating redundant information:

\ex. 
\a.\label{prencaccom}
\udrs{s',z,\anaph{x},y,v}
{children'(z)\\
girls'(v) \quad v \subseteq z\\
girls'(\anaph{x}) \quad two'(y) \quad y \subseteq \anaph{x} \quad know'(s',y)\\
\anaph{x} = v
}
\b.\label{prencfin}
\udrs{s',z,y,v}
{children'(z)\\
girls'(v) \quad v \subseteq z\\
 two'(y) \quad y \subseteq v \quad know'(s',y)
}


It is by transitivity of set inclusion that the girls are understood to belong
to the set of children established in the discourse; it does not follow
directly from the semantics of the NP as a definite superset. 

As for the zero marking, I do not propose anything special about it. It is a
standard indefinite analysis and I do not claim a disjointness semantics. I
leave it open that there arise disjointness implicatures, supported by the
existence of a formal device to signal presuppositionality, it is natural to
expect the zero-marked version to implicate disjointness. 

Now, back to \xref{conda}, repeated below:


\exg.\label{cond-w}John bir hata{\bf-yı} görür-se, makale-yi reddeder.\\
	J. a error{\bf-Acc} sees-Cond article-Acc rejects\\
	Rd.\ 1: `If John sees an error, he'll reject the article.'' (It's common
	ground that there are errors in the article.)\\
	Rd.\ 2: `An error is such that if John sees it, he'll reject the article.


\noindent Reading 1 of \xref{cond-w}:

\ex.
\a.
\udrs{j',x}{article'(x)}
\b.
\udrs{j',x}{article'(x)\\
\ccon
{\udrs{z,\anaph{Y}}{\anaph{error'(Y)}\quad  z\in \anaph{Y} \quad see'(j',z)}}
{\udrs{}{reject'(j',x)}}
{\Rightarrow}
}
\b.
\udrs{j',x,V}{article'(x)\quad error'(V) \quad in'(V,x)\\
\ccon
{\udrs{z,\anaph{Y}}{\anaph{error'(Y)}\quad  z\in \anaph{Y} \quad see'(j',z)\\
\anaph{Y}=V}}
{\udrs{}{reject'(j',x)}}
{\Rightarrow}
}
\b.
\udrs{j',x,V}{article'(x)\quad error'(V) \quad in'(V,x)\\
\ccon
{\udrs{z}{z\in V \quad see'(j',z)}}
{\udrs{}{reject'(j',x)}}
{\Rightarrow}
}

Caution: $V$ should be guaranteed to exhaust the errors in the article --
presumably by inference.


TODO: Basit hatalari gormesi onemli degil; yeter ki kritik bir hatayi gormesin.

TODO: Cogu ogrenci en az iki kitab(i) okudu. 


\subsection{Generalizing to other NP types}

\section{The proposal}

Having motivated the interpretative effect of \acc-marking on indefinites, I
turn to the grammatical mechanism that delivers that effect. This calls for a
detailed look at the noun phrase semantics in Turkish, which this section aims
to provide.

We have 4 types of nominal direct objects\footnote{My use of ``direct object''
is quite lose, not every author considers bare NPs as direct objects.} in our
scope: 1.\ bare NPs, 2.\ zero indefinites, 3.\ acc-indefinites, and 4.\
definites. I will have nothing to say in this paper about the last category. I
will explore the relative ``strength'' of the first three categories, all of
which are taken here to be a sub-type of indefiniteness. 

\subsection{Preliminary assumptions}

In standard formal semantics, a nominal expression, be it referential,
predicative or quantificational, contributes basically two things: 1.\ an $e$
type argument for a verb or predicate and 2.\ a predication (aka restriction)
of the contributed argument. In this approach, barring ``selectional
restrictions'' that are usually informally stated, verb meanings themselves are
construed as independent of the meanings of their complements. It is quite
well-known, however, that most, if not all, verb meanings are functions of the
meaning of their complements, not only at the argument-saturation level, but
also at a rather conceptual level. Here are some examples from
Turkish:\footnote{See \cttx{kratzer96} for examples from English.}

\ex.
\a. elmayı ye.
\b. parayı ye.
\b. yalanı ye.

\ex.
\a. kitabı oku.
\b. hislerimi oku.
\b. biyoloji oku.


It is important to note that none of these examples are idioms, they are quite
flexible.\footnote{Compare with:

\ex. Ayvayı ye.

where no other fruit or object would result in the same meaning.
}

As \ctnm{carlson03} observes, if we take verbs to be fundamentally denoting
eventualities (in the sense of \cttx{bach86a}), it is conceivable that one
function of nominal complements is to further specify these eventualities. As
we will see below in more detail, \ctnm{carlson03} utilizes this idea for
incorporation structures. Here, I will generalize the same idea to the entire
spectrum of verb-complement composition.

The atomic types of my ontology are $e$ for individuals, $s$ for eventualities,
$p$ for properties (or concepts) in the sense of \ctnm{mcnally98}, and $t$ for
truth values. I ignore worlds and times for the sake of simplicity.

I will assume and modify a Neo-Davidsonian verbal semantics along the lines of
\cite{kratzer96}, where external arguments are composed into the semantics by a
functional head rather than as a complement of the verb itself. I will also
assume that verb meanings have a slot for a property argument. The verb first
applies to this argument, yielding a further specified eventuality semantics.
With these assumptions, transitive verbs receive the following semantics:

\ex. \interp{\lbrac{V}{oku}} $=$ \sysm{\lam{p_p}\lam{x_e}\lam{e_s}.read'p\cnct{}x\cnct e}

Some modifications to NP semantics are also called for, so that an NP
interpretation provides the argument types required by the verb. I will assume
that every nominal comes with both a property denotation and an ordinary
restrictor predicate. For instance a nominal that is headed by the lexical noun
\emph{book} would have both a property, which I will designate as \sysm{Book'},
and an ordinary restrictor predicate \sysm{book'}. The former stands for a
prototypical (or generic, depending on the particular theory) book concept,
while the latter stands for the usual model-theoretic notion of the set of
entities that are taken to be books in the model. In this setting, a definite
description \xxref{exdefprov}{a} and a proper name \xxref{exdefprov}{b}, both
type-raised to apply to the verb,\footnote{The rationale behind this particular
choice of type-raising will appear below.} would look like:

\ex.\label{exdefprov}
\a.
\interp{\lbrac{NP$_{acc}$}{kitabı}} $=$ \sysm{\lam{v_{\smtyp{p}{\smtyp{e}{\smtyp{s}{t}}}}}\lam{e_s}.v\cnct{Book'}\cnct{(the'book')}\cnct{e}} 
\b.
\interp{\lbrac{NP$_{acc}$}{Ulysses'i}} $=$ \sysm{\lam{v_{\smtyp{p}{\smtyp{e}{\smtyp{s}{t}}}}}\lam{e_s}.v\cnct{Book'}\cnct{u'}\cnct{e}} 

External arguments are composed into the structure via a (possibly empty)
functional head, which I will simply call X.

\ex. \interp{\lbrac{X}{$\emptyset$}} $=$ \sysm{\lam{p_{\smtyp{s}{t}}}\lam{x_e}\lam{e_s}.p\cnct{e}\land agent'x\cnct{e}}

Finally, an inflection head existentially closes the event argument, also
augmenting the temporal/aspectual information.

\ex. \interp{\lbrac{Fin}{-du}} $=$ \sysm{\lam{q_{\smtyp{s}{t}}}.\exists e. q\cnct{e} \land past'e}

A simple sentence like:

\ex. John Ulysses'i okudu.

is interpreted as follows, where all merges are coupled with function application on the semantics:


\begin{tikzpicture}
\tikzset{level distance=40pt, sibling distance=10pt}
\tikzset{every tree node/.style={align=center, anchor=north}}

\Tree[.{FinP} [.{DP} John ] [.{Fin'}\\{\sysm{\exists e.read'Book'u'\cnct{e}\land agent'j'\cnct{e} \land past'e}} [.{XP}\\{\sysm{\lam{e_s}.read'u'\cnct{e}\land agent'j'\cnct{e}}} [.{NP} <John>\\{\sysm{j'}} ] [.{X'\\{\sysm{\lam{x_e}\lam{e_s}.read'u'\cnct{e}\land agent'x\cnct{e}}}} [.{VP\\{\sysm{\lam{e_s}.read'Book'u'\cnct e}}} [.{NP} Ulysses'i\\{\sysm{\lam{v_{\smtyp{p}{\smtyp{e}{\smtyp{s}{t}}}}}\lam{e_s}.}}\\{\sysm{v\cnct{Book'}\cnct{u'}\cnct{e}}} ] [.{V} oku\\{\sysm{\lam{p_p}\lam{x_e}\lam{e_s}.read'p\cnct{}x\cnct e}} ] ] [.{X} $\emptyset$\\{\sysm{\lam{p_{\smtyp{s}{t}}}\lam{x_e}\lam{e_s}.}}\\{\sysm{p\cnct{e}\land agent'x\cnct{e}}} ] ] ] [.{Fin} -du\\{\sysm{\lam{q_{\smtyp{s}{t}}}.\exists e. q\cnct{e} \land past'e}} ] ] ] 

\end{tikzpicture}

What is non-standard about the present scheme is the integration of properties
into the derivation; specifically, (i) every NP comes with a property
denotation coupled with standard restrictor denotation, and (ii) what would
appear as the normal functional semantics of a verb is computed as the result
of the application \sysm{read'Book'}, the rest is standard.

My syntactic assumptions are and will stay sketchy. I will also ignore
scrambling and other ``deviations'' from the canonical SOV order of Turkish.


\subsection{Semantic types}

In combining two semantic objects, the degrees of freedom bearing on the
outcome are threefold:

\ex.
\a. semantic types of the objects
\b. the mode of combination
\b. denotations of the objects

If we take coordination and various ``like-category deletion'' processes to be
reliable tests for semantic types in Turkish, the facts suggest that our four
categories are type-wise identical. I include here only a number of
representative examples.\footnote{Note that when a nominal requires
acc-marking, the marking is not deletable under ellipsis:

\ex. Ben Mehmet'i sevdim, Ali de Ayşe*(yi).

}

\ex.
\a. Ali öğle yemeğinde bazen cips, bazen küçük bir elma, bazen de evden getirdiği yemeği yiyor. 
\b. Doğumgününde ben Ayşe'ye eski bir kolye verdim, Ali de babannesinden kalan yüzüğü. 

\ex.
\a. Ahmet ve Ayşe bana geldiklerinde, masanın üzerinde bir dolu kitap vardı.
\b.\label{typeell} Ayşeye bir romanı verdim, Ahmet'e de bir kadeh viski.

TODO: Explain the examples

On account of these observations, I assume that all the four types of nominal
objects are of, or can be shifted to, the same semantic type. I also take the
verb semantics to be constant. This decision is justified in Turkish for the
following reason: The verbs that take optional accusative are not a small
special class, it is quite common among Turkish verbs, and this contrasts with
the typical example of this kind of verbal ambiguity, namely Greenlandic
\cttxp{geenhoven98}.

Taking the possibly shifted types of objects and their verbs constant also
fixes the mode of combination. I simply take the object-verb combination to be
function application. In this regime, objects are functions from transitive
verb meanings to intransitive verb meanings (or verb phrase meanings):

\ex.
\a. verb type: \smtyp{e}{\smtyp{s}{t}} 
\b. object type: \smtyp{\smtyp{e}{\smtyp{s}{t}}}{\smtyp{s}{t}} 

\subsection{Bare objects}

The syntax of bare objects has received considerable attention in the
literature, where the central discussion was around the issue of whether the
bare objects are phrasal (pseudo-incorporated) or not (head incorporated) (see,
among others,
\cttx{erguvanli79,nilsson85,knecht86,kornfilt03,aydemir04,kechriotis06,ozturk05,ozturk09,gracaninissever11}).
Here I am not concerned with the details of syntactic issues. I would, however,
like to distinguish three cases:

\ex.
\label{exbalik1}
\a. Ahmet balık tuttu.
\b. *Ahmet levreğ(i) balık tuttu.

In \xxref{exbalik1}{a} we have a bare object composed with the verb giving a
stereotypical verb phrase (= `fishing'). The ungrammaticality of
\xxref{exbalik1}{b} indicates that the (pseudo-)incorporation operation
saturates the argument slot of the verb \trtx{tut}{catch}, leaving no room for
a direct object \trtx{levreğ(i)}{sea bass(-Acc)}, with or without the
accusative.

\ex.\label{exbalik2} 
\a.
Ahmet balık aldı.
\b.
*Ahmet levreği balık aldı.

In \xxref{exbalik2}{a} bare object-verb composition yields a non-stereotypical
verb phrase, and \xxref{exbalik2}{b} likewise indicates that the direct object
is saturated by this composition.

\ex.
\label{exbalik3}
Ahmet (konuşmasında) Mehmet'i hedef aldı.

\xref{exbalik3} shows the composition of the verb we had in the previous two
examples with the bare object \trtx{hedef}{target} constructing the complex
verb \trtx{hedef al}{target}, which is still transitive after the composition,
as it can still combine with the acc-marked direct object
\trtx{Mehmet'i}{M.-Acc}.

In what follows I will unite the cases exemplified in \xref{exbalik1} and
\xref{exbalik2} under the name ``bare object'', and leave out the complex verb
structures exemplified in \xref{exbalik3}. I cannot see any difference between
such verbs and lexical transitives that would concern us in this paper.

I first establish at an intuitive level that bare objects are predicative
rather than referential. The argument is based on an observation about
kind-level interpretation in Turkish.\footnote{As \cttx[15]{krifkaetal95}
observe, all semantic distinctions we are interested in this paper, like
specific/non-specific, predicative/referential etc.\ apply at kind-level as
well.} The reason I employ a kind-level example is that the predicative nature
of bare objects is most perspicuous at this level. Take the minimal pair:

\ex.\label{exzimba}
\a. Amerikalılar 19. yüzyılda  zımbayı icat etti.
\b. Amerikalılar 19. yüzyılda zımba icat etti.

In \xxref{exzimba}{a} there is a direct reference to the kind \emph{stapler}, while \xxref{exzimba}{b} carries a kind-level predication, resulting in what \ctnm{krifkaetal95} calls a ``taxonomic'' reading.

One way to integrate this predicative meaning into the verb phrase semantics
would be to use \ctnms{chungladusaw04} Restrict. This would additionally call
for an existential closure operation, since Restrict does not saturate argument
slots, while Turkish bare objects do. Assuming that kinds are atoms of type
$k$, a Restrict + EC derivation would look like:

\ex.

\begin{tikzpicture}
\tikzset{level distance=40pt, sibling distance=10pt}
\tikzset{every tree node/.style={align=center, anchor=north}}

\Tree[.{EC}\\{\sysm{\lam{e_s}\exists x_k.invent'x\cnct{e} \land stapler'x}} [.{Restrict}\\{\sysm{\lam{x_k}\lam{e_s}.invent'x\cnct{e} \land stapler'x}} [.{NP}\\{\sysm{\lam{x_k}.stapler'x}} ] [.{V}\\{\sysm{\lam{x_k}\lam{e_s}.invent'x\cnct{e}}} ] ] ]

\end{tikzpicture}

The same mechanism would give a parallel interpretation for a non-kind example like \trtx{kitap oku}{`book read'}:

\ex.
\interp{\lbrac{VP}{kitap oku}} $=$ \sysm{\lam{e_s}\exists x_e.read'x\cnct{e} \land book'x}


The problem with this solution lies in the existential quantification of the
object argument of the verb, directly delivered via semantics. Existential
quantification always affords (possibly modally subordinated) discourse
reference. But, as first observed by \ctnm{erguvanli79}, Turkish bare objects
do not readily contribute discourse referents and usually result in
number-neutral interpretations.

\ex. ??Bugünlerde kitap okuyorum, bitirince sana vereceğim.

\ex. Kampın ilk gününde çocuklar sahilde balık tuttular. Maalesef çoğu tek balık bile yakalayamadı.

(activity with a perfective aspect)


\ex.
\a. Ahmet keman çalıyor.
\b.  Yeni mi?


\ex.
\a. Dün bütün gece şiir yazdım.
\b. Kaç tane yazdın?
\b. Sıfır!/Çok!

As we will see in the next section, this behavior sharply contrasts with zero
indefinites, where discourse reference is guaranteed.

There are cases where bare objects seem to contribute discourse referents,
however. This possibility appears to be dependent on the aspectual class and
conceptual details of the verb. For instance, it seems rather impossible to
think of an onion chopping process that does not involve some amount of onions,
and it is also expected that the use of a bare object expression may foreground
the amount or kind of onions involved.  The point is that such contributions to
the discourse model by bare objects are not semantically driven, but possible,
thanks to some inferential processes that we do not examine here.

The present paper proposes that what is contributed by a bare object
construction is the instantiation of an eventuality type rather than an
individual discourse referent. In this proposal the bare object is a sort of
event modifier in a fashion very similar to \cttx{carlson03}. In this respect I
take bare objects to be basically denoting properties, totally devoid of
quantificational force and not contributing discourse referents into the
discourse model. There is an apparent conflict in this list of requirements. On
one hand we do not want a discourse referent to be contributed, on the other
hand we know from above that the bare object saturates the direct object slot
of its verb. I deal with this conflict by stipulating that bare objects
contribute a semantically inert argument to the verb, which I will represent as
\sysm{null'}. A type-raised bare object takes the following interpretation:

\ex.\label{exintbare}
\interp{\lbrac{NP}{kitap}} $=$ \sysm{\lam{v_{\smtyp{p}{\smtyp{e}{\smtyp{s}{t}}}}}\lam{e_s}.v\cnct{}Book'null'e}

The derivation of the VP semantics is as follows:

\ex.\label{exdervpbare}
\begin{tabular}[t]{l@{\hspace{0.2em}}l}
\interp{\lbrac{VP}{kitap oku}} &$=$  (\sysm{\lam{v_{\smtyp{p}{\smtyp{e}{\smtyp{s}{t}}}}}\lam{e_s}.v\cnct{Book'}null'e}) (\sysm{\lam{p_p}\lam{x_e}\lam{e_s}.read'p\cnct{}x\cnct e})\\
&$=$ \sysm{\lam{e_s}.read'Book'null'e} \\
\end{tabular}

Integrating the subject via the X head and closing the event argument with
inflection head yields the final result through the following steps:

\ex.\label{exbareresult} 
\a.
\begin{tabular}[t]{l@{\hspace{0.2em}}l}
\interp{\lbrac{\ubar{X}}{\lbrac{VP}{kitap oku} \lbrac{X}{$\emptyset$}}} & \\
&\kern -60pt $=$  \sysm{(\lam{p_{\smtyp{s}{t}}}\lam{x_e}\lam{e_s}.p\cnct{e}\land agent'x\cnct{e})(\sysm{\lam{e_s}.read'\cnct{Book'}\cnct{null'}\cnct{e}})}\\
&\kern -60pt $=$  \sysm{\lam{x_e}\lam{e_s}.read'\cnct{Book'}\cnct{null'}\cnct{e}\land agent'x\cnct{e}}
\end{tabular}
\b.
\begin{tabular}[t]{l@{\hspace{0.2em}}l}
\interp{\lbrac{XP}{\lbrac{NP}{John} \lbrac{\ubar{X}}{kitap oku $\emptyset$}}} & \\
&\kern -40pt  $=$ \sysm{(\lam{x_e}\lam{e_s}.read'\cnct{Book'}\cnct{null'}\cnct{e}\land agent'x\cnct{e})j'}\\
&\kern -40pt  $=$ \sysm{\lam{e_s}.read'Book'null'e \land agent'j'\cnct{e}}\\
\end{tabular}
\b.
\begin{tabular}[t]{l@{\hspace{0.2em}}l}
\interp{\lbrac{\ubar{Fin}}{\lbrac{XP}{John kitap oku $\emptyset$} \lbrac{Fin}{-du}}} & \\
& \kern -120pt $=$ \sysm{(\lam{q_{\smtyp{s}{t}}}.\exists e. q\cnct{e} \land past'e)(\lam{e_s}.read'Book'null'e \land agent'j'e)}\\
& \kern -120pt $=$ \sysm{\exists e.read'Book'null'e \land agent'j'e \land past'e}\\
\end{tabular}

\textcolor{red}{ The reader might wonder why I choose a little complex
mechanics to model the argument decrement behavior of bare objects, wouldn't a
valence-changing operation suffice. The reason is that bare objects show some
degree of syntactic flexibility, therefore they live in syntax rather than
morphology.}

With this formulation we model the following empirical facts:

\ex.
\a. Bare objects yield number neutral interpretations.
\b. They do not readily provide discourse referents.
\b. They saturate the internal argument of the verb. 
\b. They foreground the type of the eventuality rather than the internal argument.


\subsection{Zero indefinites}

I will argue that the only difference between a bare object and a zero indefinite is the discourse referent contributed by the latter. The common denominator of the two NP types is that their predications are at the property (or concept) level. They are contextually unbound in the sense that the eventuality in the bare object case and the discourse referent in the zero indefinite case carries no information beyond what is available at the general concept level. In exactly this respect they will differ from acc-indefinites.

%I will assume that the expression \emph{bir} is ambiguous between the numeral one and the indefinite determiner, and likewise for other numerals. It may be possible to collapse the two interpretations into one, but I will not attempt this here.

The function of the indefinite determiner is to carve out individuals from properties and predicates. The number semantics carried by the determiner determines the cardinality of these individuals. A zero indefinite has the following interpretation:

\ex.
\interp{\lbrac{DP}{\lbrac{D}{bir} \lbrac{NP}{kitap}}} =
\sysm{
	\lam{v_{\smtyp{p}{\smtyp{e}{\smtyp{s}{t}}}}}\lam{e_s}.
	v\cnct{Book'}(sk'(
					   \lam{x}.ins'Book'x \land |x|=1))e}

The function \sysm{sk'} is a non-parametrized Skolem function and when applied to a property, returns a prototypical/arbitrary instance of that property.

The most crucial aspect of this formulation is that there is no way the sentential or utterance context to affect the interpretation of the book instance. Different from the bare object case, this time we have an individual, albeit a generic one, contributed to the discourse model.

\ex.\label{exbolbol}
\a. Bugünlerde bol bol kitap okuyorum.
\b. *Bugünlerde bol bol bir kitap okuyorum.

\xref{exbolbol} shows that with zero indefinites individual reading is foregrounded where the eventuality denotation is not available for adverbial modification (see also \cttx{aydemir04}).



\subsection{Acc-indefinites}

Although syntactic issues are not my main concern, a few points regarding the syntactic status of Acc-indefinites are in order.

I think the accusative case alternation on indefinite objects is not a clear case of Differential Object Marking, which, following \cite{lopez12}, I take to be proposing a direct morphology/semantics relation between case and specificity (or some other interpretative category) and that this relation operates on the basis of a definiteness scale. I follow the Turkish literature in claiming that there is no direct morphology/semantics relation in accusative case in Turkish. Accusative is rather a reflex of syntactic organization, which itself has an effect on the interpretation. A common denominator of this literature is that the overt case is a reflection of syntactic position and this position itself is responsible for the interpretative differences between having and not having overt case. I follow this idea here. Turkish nominal expressions do not carry overt case when they remain in a domain projected by the verb, and along the logic of \ctnm{diesing92} they thereby receive a sort of non-referential reading in this domain.


Another reason to think that the presence of Acc-marking is a syntactic requirement, observe that possessive DPs obligatorily receive case-marking, regardless of their interpretative properties.

\ex. Polis Ali'nin bir fotoğrafı*(nı) arıyor.


Furthermore, as in the set of languages discussed by \cite{lopez12}, objects in small clause constructions and objects that control PRO obligatorily receive overt case:

\ex.
\a. Bir öğrenci*(yi) zeki bulabilirim.
\b. Gözetmenlik yapmaya bir öğrenci*(yi) ikna edeceğim. 


Now we can turn to the semantics of acc-marked indefinites.

The account I will provide here concurs with the proposal of \ctnm{enc91} in taking the interpretative effect of \acc-marking to be related to definiteness. I will however diverge from \ctnm{enc91} in the exact nature of this definiteness, as her account misses the ``strength`` of \acc-indefinites, demonstrated above in Section~\ref{scstrong}.

Semantics of definiteness is usually associated with (subsets of) the following notions: uniqueness, exhaustivity, identifiability, existence and familiarity (TODO:refs). Some of these dimensions may apply both at the referent and restrictor level. I will argue that for an \acc-indefinite, the relevant notions are familiarity and existence, and that familiarity and existence applies at the restrictor level only. In this picture referents are indefinite and the restrictor is definite, akin to \ctnm{enc91}. For \ctnm{enc91} definiteness involves the existence of a familiar superset of the restrictor, while in the present account, what is definite is the restrictor itself, where what is meant by definiteness is familiarity and existence. 

\ex. \a. Ali bana bir koli kitap bırakmıştı.
\b. Geçen hafta iki romanı okudum.
\b. Geçen hafta iki roman okudum.

To better assess the function of \acc-marking, consider further the following minimal pair:

\ex.
\a. Bunların yerine geçen hafta iki roman okudum.
\b. Bunların yerine geçen hafta iki romanı okudum.

Here the speaker does not have any commitment to the identifiability of the
novels, the hearer may well have no idea about the identity of the novels.
There also exists no commitment on the size of the restrictor \sysm{novel'},
there may be just two or more than two novels, no choice is indicated.
Therefore the \acc-indefinite does not indicate any of identifiability,
uniqueness or exhaustivity. Incidentally, the \acc-indefinite also does not
indicate non-exhaustivity, which might have been suggested by an error in one
of \ctnms{enc91}  glosses (TODO:ex), it simply has no say on the exhaustivity
dimension.


PERHAPS: drt formulation'inin son kismini buraya tasi.


\ex. \a.Yarin toplantiya bir moderator getirecegim.
\b. \#Yarin toplantiya bir moderatoru getirecegim.

TODO: Bir Nazim Hikmet siirine yapilmis ilk beste: Bana Bak Hey Avanak

\section{Empty item}


To summarize:

\ex.
\a. Bare objects are contextually unbound eventuality modifiers.
\b. Zero-indefinites are contextually unbound individual contributors.
\b. Acc-indefinites are contextually bound individual contributors with presupposed restrictors.






TODO: Bir soru(yu) *bile* sormadim.

TODO: Give a list of interpretative/scopal/IS-related effects in the first section.

TODO: Acc marking does not necessarily have functional readings (contra KvH).

TODO: negative verbs require the marker: engelle. durdur, (relate this discussion to verbs of creation).

TODO: Dün ikişer görevli her hücreyi ziyaret etmiş.

TODO: strong quans are scope rigid but acc-marked indefs are not.

TODO: Evidence that Possessives cannot stay VP internal.

TODO: Acc is not related to scope, it marks an anaphoric dependency of the restrictor.

TODO: Ali projede iki yabanci(yi) calistiriyor.

TODO: Ali annesine bir ogrenci*(yi) sikayet etti.

TODO: Zero do not introduce discourse marker;

Ahmet bana bir cocuk gosterdi; annesini kaybetmis yeni. 

Annesine *Bir cocuk gostermeye gitti.

TODO: The domain that contains bare objects is sensitive (or closely related) to information structure. Focus domain can expand to left, but it cannot contract to right excluding a bare object.

TODO: 
Rize'de balik yiyeceksin.
Baligi Rize'de yiyeceksin. 
Hamsiyi bugulama yapacaksin.
Hamsiyi buzluga atacaksin. 
Buzluga hamsi atacaksin.

Ruslar zimbayi icat etmis.
Ruslar zimba icat etmis.

Zimba is a kind predicate (true or false of kinds). It is then <s,t>. But zimbayi is referential, it refers to the kind zimba, therefore it is of type s.

Zimbayi bana uzat.

This is reference to an ordinary individual.

Conclusion: there is a merger that merges and <x,t> with a transitive; it saturates the object argument.


It is also important to notice that the zero versus case-marked distinction crosscuts nominal types. I give examples somewhere else.

I start with the following key observation:

\ex.\label{ex:doktor} 
\a. Ali doktor.
\b. Ali bir doktor.

Basic assumptions: I follow \cite[p.\ 64]{krifkaetal95} and commit to a basic type of kinds alongside with standard individuals, I will have types \sysm{e} and \sysm{k}. The relation between a kind and its instances will be handled by \sysm{Ins} -- this is Carlson's R, Krifkaetal just remove the stages. The relation is many-to-many, a given object may be an instance of more than one kind and, more obviously, a kind may have more than one instance. Predicates can be true of standard or kind individuals. For instance, the predicate \textit{is invented by Americans} ranges over kind individuals. One alternative would be to take kind-referring NPs to denote the sum of all the individuals that the nominal applies to (see \cite{krifkaetal95} for references).

If I take the genericbook route, which is to take a kind as basic and define the associated property as (lambda x: R(x,k)), then I need genericbooks `We leave it open as to whether every predicate has a corresponding kind'.

Syntax: case filter is fulfilled under adjacency, so bare NPs are also assigned case, but accusative indicates that the NP is not in the adjacent position. Bir is the manifestation of the Carlsonian Realization Relation. When out of the existential closure domain of the immediately preverbal slot, nominals are forced to get argumental/referential interpretation.

\begin{comment}
Krifkaetal
* p87 incorporation and kind-reference.
\end{comment}

Our dimensions are count/mass, generic/kind/weak indefinite/definite.

FACT 1: Kinds are referred to by singular bare NPs, regardless of count/mass distinction:

\ex.
\a. Leopar(?*lar)a Anadolu'da rastlanmaz.
\b. Kemençe(*ler)nin anavatanı Görele'dir.
\b. Kayısı(*lar) Malatya'da ilk defa 19. Yüzyıl'da yetiştirildi. 
\b. Kayısı(*lar) Malatya'nın sembolüdür.
\b. Zımba(*lar) 19. yy'da ABD'de icat edildi.

are all kind predications in the sense of \cttx{krifkaetal95}.


\ex.
\a. Son yillarda adada köpek(*ler) bulunmuyor.



It is also important to observe that a kind-referring NP at the direct object position requires an accusative case:

\ex.
\a. Amerikalılar zımbayı icat etti.
\b. Amerkalılar zımba icat etti.
\b. Amerkalılar yeni bir zımba icat etti.

The second one means that Americans invented a new kind of stapler. The interpretation is still kind-referring. Therefore kind/non-kind distinction is orthogonal to acc-marking (see \cttxp[p.\ 15]{krifkaetal95})

The question here is whether the kind denotation is a lexical property or these NPs are just definites -- remember that Turkish does not have a definite article and case-marked NPs in argument positions get interpreted as definite. I will now try to find a principled reason to choose one type of interpretation as basic and the other derived.

It appears that kind reference is not uniquely possible by bare NPs, bare plurals seem also fine with denoting kinds, especially with animals. The following pair is apparently synonymous:\footnote{`Motosiklet tehlikelidir' gets 1010 Google hits while `Motosikletler tehlikelidir' gets 162 -- Thu 25 Feb 2021 09:27:48}


\ex.
\a. Aslan etobur bir hayvandır.
\b. Aslanlar etobur hayvanlardır.
\b. Timsah olan bir suda yüzülmez.
\b. Timsahların olduğu bir suda yüzülmez.
\b. *Timsahın olduğu bir suda yüzülmez.
\b. Yılanlar (Latince: Serpentes), Pullular (Latince: Squamata) takımına ait uzun, ayaksız etçil sürüngenlerdir. (Vikipedi -- Thu 25 Feb 2021 09:36:50 AM +03)

The plurals here, however, seem to refer to the families of kinds rather than a unique kind:

\ex.
\a. Çobanaldatan (Caprimulgus europaeus), Çobanaldatanlar (Caprimulgiformes) takımının çobanaldatangiller (Caprimulgidae) familyasına ait Caprimulgus cinsinden bir gece kuşu türü.
\b. Gitar, parmakla veya pena ile çalınan, esasen sekiz şekline benzeyen, yan kısımları oval, sap üzerinde ses perdeleri olan, telli bir çalgı türü. Gitarlar genelde altı tellidir ve farklı çeşitlerdeki ağaç türlerinden yapılabilirler. Gitar neredeyse her türlü müzik türünde kullanılan bir müzik aletidir. 


Nationalities on the other hand are always plural:

\ex.
\a. İtalyan*(lar) akşamları şarap içer.

I conclude that bare singular NPs denote kinds in Turkish, bare plurals are just plural kind reference, resorted to when the sub-kinds of a kind are at issue.

Bare plural NPs are for generic reference, plural definite descriptions and weak indefinite reference.

\ex.
\a. Köpekler bahçeye girmiş. (plural definite)
\b. Bahçeye köpekler girmiş. (weak indefinite)
\b. Köpekler çocukları sever. (generic)

I assume that generic statements are also possible with kind terms TODO:check Generic Book.

One interesting aspect of the Turkish nominal system is that weak indefinite interpretation is also possible with bare NPs: 

\ex.
\a. Bahçeye köpek girmiş. (weak indefinite)
\b. Köpek bahçeye girmiş. (definite)
\b. Muz yemek mideme iyi geldi. (weak indefinite)
\b. Dolapta muz kalmamış. (weak indefinite)

here number is not implied, at least one dog is enough to make the proposition true.

There is one aspect of the NP that differs with respect to its location. What is that aspect?

Kind predicates:
Faks kalmadi/yok.
Yakinda Hamsi tarihe karisacak.



The effect of the marker is related to definiteness.

\intx{bir} is an operator, can be poly-typed or not; it applies to a bare NP which is a kind term, and carves out a brand-new individual untied to the context. The same operator app		 lies to an \acc-marked NP which is in a sense definite. In the absence of \intx{bir} the NP becomes standard definite due to maximality. In the presence of \intx{bir} a partitivity is indicated, while the restrictor is still definite. What do I do with scope then? Either schwarzschild -- for some reason I do not like that idea, or some other explanation. Other possibility is this: in the absence of an operator like negation, the two interpretations collapse. But still we need an explanation for why the scope is flexible for \acc-marked indefinites. This I can explain from the other direction. \zero-marked indefinites have an adjacency condition, they are the exceptional ones, not the \acc-marked ones. This in turn might be motivated by information structural concerns.    



I claim that Turkish data can be captured in a model where syntactic saturation is isomorphic to semantic saturation and where the semantic composition is restricted to function application. The proposed model is most closely related to the combinatory varieties of categorial grammar (\cite{steedman00b,jacobson99}). 

I leave it to a later occasion the discussion of whether the wide-scope
reading (Rd.\ 2) comes from a raising mechanism or via restriction to a
singleton domain ala \ctnm{schwarzschild02}.

A Diesingian account would have that \acc-marking is a trigger (or indicator) of
raising out of a verbal domain associated with existential closure. Such objects
would require their own quantificational force. On the other hand NPs that
already carry quantificational force will be forced out of the existential
closure domain, thereby obligatorily receiving the \acc-marker.


\begin{uquestion}
Why anti-uniqueness effect is not observed for IPV acc-NPs.
\end{uquestion}

\begin{comment}

# Ozturk 2005:

main claim: no agree relation from a higher projection in turkish; case is a local phenomenon, assigned in the local projection; case and referentiality are united, therefore no need for a hierarchical configurational phrase structure.
	- Turkish demonstratives, like in Japanese, do not close projections: John'un bu kitabi.
	- She takes 'kirmizi bu kitap' = 'this red book', which is quite controversial.
	- 19-21: why turkish does not have a definite or indefinite determiner.
	- Bare caseless objects are phrasal.
	- p47: they are not arguments; they do not passivize -- they only yield impersonal passives;  they form complex unergative predicates with the verb.
	- p57-8: Doktor hastayi muayene mi etti?/\* Doktor hasta muayene mi etti?
	- p61: Turkish bare NPs are [+pred,-arg], when they are visible for case, they are type-shifted to +arg, otherwise they stay pred and form a complex verb.
	- Case is a type-shifter, it shifts to kinds and definites but not to E.
	- N tane is extractable; 69-70. 
	- WHat unites definties, kinds and spec. indefs. is their being singletons, case can apply to them all.
		- She dismisses Aydemir data with controversial data.
	- She dismisses the possibility of anaphora to bir N's as accommodation.


# Diesing 1992:

* As Diesing has this system for subjects (VP-Internal Subject Hypothesis) it
should apply to objects as well.

* At LF vp int. ext subjects are mapped to quantificational structures.

* Following Heim, restrictive clause -- nuclear scope	partition (has roots in
theme-rheme, topic-comment 	(p.\ 5).

* Existential closure is a last resort operation that prevents unbound variables.

* Strong/weak: Partee 88 few and many; diesing also maintains that there are two
different types of indefinites; presuppositional versus cardinal determiners of
Milsark 1974.

* There is a relation between presuppositionality and obligatoriness of
QR.

* Island constraints are sensitive to LF (p. 13).

* Verbs interacting with the presuppositionality of their objects.

* Individual versus stage level predicates.

> `is ready' might be a counterexample to ind versus stage.

* Subjects of ind-level preds can appear only in the restrictor; subs of
stage-level can appear in both p19.

* The exact position of the VP-internal subject is not important, as long as it is
in the VP domain. p20.

* Stage versus individual is raising versus control.

* Adverbials as VP boundary refs p31.

# Chung and Ladusaw 2004

* Restrict does not saturate an argument, but shifts its lambda to right before the event lambda. 
* A predicate must be fully saturated at the event-level (=when there is nothing left to combine but the inflectional head).
* if any unsaturated argument left at e-level, EC saturates it -- EC may apply sooner.
* negation is interpreted above the event level -- they take this to be the standard view p13.
* ex 29, p13 has an unsaturated argument below event EC -- there is an explanation on the following page that says this does not count as unsaturated -- I don't understand why.
* Specify is CF + App, the choice function is existentially closed non-deterministically (Reinhart, Winter, contra Kratzer and Matthewson)  
* Morphosyntax signals whether to compose with Restrict or Specify.
* he is restricted to subjects.
* some claim tetahi is `the'+`one'.
* Both Maori determiners he and tetahi can be narrow wrt Cond, Neg (see also p.52), Ques, Quan -- for such contexts they are interchangeable p40.
* tetahi can be wide wrt operators but he cannot.
* They return to specificity; external subjects (ind-level, transitive, unergative verbs) must be specific, and only tehati headed NPs can be specific (as they become e type by CF). 
* tehati headed NPs are more prominent in discourse.
* They make lambda shift afer Restrict non-deterministic p109


# Farkas and de Swart 2003

* They distinguish full-fledged narrowest scope (scopal inertia) indefinites from incorporated ones, in contrast to van Geenhoven 1998, who takes the two classes to be identical.
* p104-5 full-fledged indefs cannot scope under negation; this cannot be due to egy (indefinite article) being a PPI, since complex NPs headed by egy can scope under negation. I do not understand the explanation that they base on this fact.  
* TERM: dependent (necessarily co-varying); roofed (require a commanding operator, like NPIs).
* They distinguish between thematic arguments and discourse referents. The former come with predicative categories like V and N, the latter are contributed by determiners.
* In _Mary is a doctor_, _a_ is expletive (de Swart 2001).
* p37, ``The scopal properties of a discourse referent introducing nominal are determined by the item responsible for the introduction of the discourse referent.''

* 51-2 inflection -- like plural in their case -- requires local accommodation, while lexical determiners are more flexible. 

* Plurarlity is an indicator of non-atomicity.

# Lopez 2012:

* syntactic structure limits the availability of modes of combination; its effect is not direct as in diesing92.

\end{comment}

\begin{udefinition}
A nominal is weak if its dref gets created at the point it is first evaluated; and there exists no dref that is identical to it in the previous discourse as a variable or as a member of an accessible set. Except accidental coreference.
\end{udefinition}


\section{Conclusion}

The paper argued that the accusative marker in Turkish marks definiteness minus
uniqueness.

Turkish is important as it shows us that various components of the grammar of quantification and reference can have separate morphological reflexes, and thereby provides evidence for a case of compositionality in quantification \cttxp{szabolcsi10}.

\appendix

\section{The formalism}

As the semantic representation language I use DRT. I use a non-boxed notation, which is a little harder to read but saves space.  

We do not have any simple DRSs; each sentence, regardless of being quantificational or not, gets interpreted as a tripartite structure. 


\ex. 
\a. Mary sleeps.
\b. \sysm{[x:mary'x,sleeps'x]}


VP interpretation:

\ex. sleeps \sysm{:= \lambda x.[: sleeps'x]}

\ex. Mary   \sysm{:= \lambda p.[x: x = m']\langle \forall \rangle p\, m'}

\ex.\a. Mary sleeps.
\b. \sysm{[x: x = m']\langle \forall \rangle [: sleeps'm']} 


The generalized quantifier interpretation for the proper name \intx{Mary} puts into a higher DRS

\ex. every woman \sysm{:= \lambda p.[x: woman'x]\langle\forall\rangle p x} 

\ex.\a. Every woman sleeps.
\b. \sysm{[x: woman'x]\langle\forall\rangle [: sleeps'x]}

Every sentence applies to a DRS and inserts its condition into it. I model this with a two-place mood operator which applies to a sentence  and a DRS representing the current state of the discourse, and updates the DRS with the sentence. 

\ex. UPDATE$_{DEC}$ \sysm{:= \lambda s\lambda k. k\oplus s} 

I do not give the algorithm for $\oplus$; informally what it does is to append the complex condition to its right to the conditions on the top most level at the DRS to its left. It also resolves all the presuppositions present in the complex condition argument. I also do not give the resolution algorithm.



\section{The fragment}


\setlength{\bibsep}{0pt}
\bibliography{ozge}
\bibliographystyle{apalike}
\end{document}

* acc marking has syntactic basis: 
	Şüphelinin bir fotoğrafın*(-ı) gördü.
	Şüpheliye ait bir fotoğraf(-ı) gördü.

* case and referentiality.

* In generative grammar, two notions are relevant:
	- visibility for theta-role assignment
	- referentiality -- ability to appear at argument position; or type-shifting to individual denotation.
	- theta-role assignment may be seen as a theory-internal issue; but referentiality is cross-theories.
	- I stick to a DRT approach to referentiality -- namely contributing a discourse referent; modally subordinated representations included.  - This understanding of referentiality is not restricted to individual denotations. GQ's are also referential since they contribute discourse referents at their local domain, which appear as arguments of predicates. see also Gomeshi 1999.

* case and argumenthood/referentiality:
	- Ben seni bekçi sandım.
		is ambiguous btw equative and attributive. In the former, _watchman_ has individual denotation; but where does it get its case?  


* Zor bir soruyu sormaliydi.
* determiner issue: translation, besides constraints on placement and stress Kornfilt-review.  
* pseudo-incoporation in dative:
	- Ali kendini kitaba verdi. -- or is this a kind term?

* Neden sinirlendi -- cunku ona kolay bir soruyu sormadim.

## Auxiliary notes 

* Öztürk 2005: main claim: no agree relation from a higher projection in turkish; case is a local phenomenon, assigned in the local projection; case and referentiality are united, therefore no need for a hierarchical configurational phrase structure.
	- Turkish demonstratives, like in Japanese, do not close projections: John'un bu kitabi.
	- She takes 'kirmizi bu kitap' = 'this red book', which is quite controversial.
	- 19-21: why turkish does not have a definite or indefinite determiner.
	- Bare caseless objects are phrasal.
	- p47: they are not arguments; they do not passivize -- they only yield impersonal passives;  they form complex unergative predicates with the verb.
	- p57-8: Doktor hastayi muayene mi etti?/\* Doktor hasta muayene mi etti?
	- p61: Turkish bare NPs are [+pred,-arg], when they are visible for case, they are type-shifted to +arg, otherwise they stay pred and form a complex verb.
	- Case is a type-shifter, it shifts to kinds and definites but not to E.
	- N tane is extractable; 69-70. 
	- WHat unites definties, kinds and spec. indefs. is their being singletons, case can apply to them all.
	- She dismisses Aydemir data with controversial data.
	- She dismisses the possibility of anaphora to bir N's as accommodation.


* referentiality comes from D; case comes from agree via TP v etc.

* Longobardi's Determiner Phrase Hypothesis: D shifts N to an individual type.

* Taylan 86: da mi bile can infix in bare NP constructions.

* On weak-strong; keenan and stavei, keenan 87, and barwise cooper 81. -- see
  Keenan 2003 first.

* de Hoop 96: agrees with Fodor and Sag; but does not think indefs are ambiguous. The particular interpretation is condition via syntax: strong indefs are referential, weak indefs are quantificational.

* de Hoop96: enc is wrong in collapsing referentiality and partitivity.

* Erguvanli 84: non-derived adverbials, like hizli, are immediately preverbal.

* heusingerkornfilt17: Meyvelerden kirmizi olan(lar)\*(i) yedim. this is also
  ungram, you do not need sI(n) for unrammaticality. But see is different,
  meyvelerden kirmizi olanlar gordum.
