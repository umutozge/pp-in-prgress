\documentclass[11pt,a4paper]{article}
%\documentclass[glov3,smallextended,nospthms,natbib]{svjour3}

\usepackage{geometry}

\usepackage{stmaryrd}
\usepackage[T1]{fontenc}
\usepackage{umut,umuttr}
\usepackage{uling}
\usepackage{usynsem}
\usepackage{udrt}
\usepackage{utheorem}
\usepackage{mathptmx}
\usepackage{hyperref}
	\usepackage{xcolor}
\hypersetup{colorlinks=false,linkbordercolor=red,citecolor=gray,linkcolor=gray,pdfborderstyle={/S/U/W 0}}
\usepackage{fancyvrb}
\usepackage{comment}

\usepackage{microtype}

\usepackage{tikz-qtree}
\usepackage[normalem]{ulem}

%linguex adjustment
\setlength{\Exlabelsep}{0.5em}

% paper problem
%\setlength{\hoffset}{50pt}
%\setlength{\voffset}{40pt}


\newcommand{\encspec}{Enç-specific}



\title{On the strength of indefinites: The view from Turkish}

\author{Umut \"Ozge}

% \institute{Umut \"Ozge
% 			\at Middle East Technical University \& University of Cologne
%  			\\ \email{u}}
% 
% \titlerunning{Strength of indefinites}
% \journalname{}

\date{\today -- PLEASE DO NOT CITE}


\begin{document}
\maketitle

\begin{abstract}
The paper claims that the \acc-marker in Turkish is an indicator of definiteness
minus uniqueness.
\end{abstract}

\section{Introduction}

Representationalism in discourse semantics
\cttx{karttunen76,webber78,kamp81,heim82} ...

this naturally brings the burden/issue of how to manage the referents


Start with some general remarks on indefiniteness; then introduce the notion of
``strong'' indefiniteness; and then go on with Turkish, by saying that Turkish
has some interesting data that is relevant to this debate.

\exg. {\label{para-bare}John} {\bf kitap} {okudu.} \\
	{J.} {\bf book} {read}\\
`John did book-reading.'

\exg. {\label{para-def}John} {\bf kitab-{\i}}  {okudu.} \\
	{J.} {\bf book-Acc}	{read}\\
`John read the book.'

\exg. {\label{para-indef}John} {\bf bir} {\bf kitap} {okudu.}\\
	{J.} {\bf a}  {\bf book} {read}\\
`John read a book.'

\exg.{\label{para-acc}John} {\bf bir} {\bf kitab-{\i}} {okudu.}\\
	{J.} {\bf a} {\bf book-Acc} {read} \\
`John read a book.' (``strong'')


% TODO: cite vonhuesingerkornfilt in discussing that the alternation is not unconditionally applicable.


In this paper we are interested in \xref{para-acc}. Two facts make
this form interesting. One is that the \acc-marker is strongly
associated with definiteness in Turkish. This is most apparent in the
minimal pair \xref{para-bare} \versus\ \xref{para-def}.
\cttx{geurts10} associates specificity with definiteness, and this is
what we will do below. Furthermore, for noun phrases that are usually
considered definite---in the sense of displaying definiteness
effect---the marker is obligatory. These constructions are:

\ex.
\a. proper nouns;
\b. pronouns and demonstratives;
\b. definite descriptions;
\b. ``strong'' DPs; % TODO: give a list
\b. derived nominals;
\b. genitive possessive constructions.\footnote{Genitive-possessives
are the only type of construction in the list for which an indefinite
interpretation is possible. I leave the analysis for this construction
for future work. See \cttx{coppock13} for discussion that might be
relevant for the Turkish facts as well.}

The interesting case here is genitive possessive constructions. They require the
marker but they can be indefinite. The others are definite. Therefore, with the
exception of the genitive possessive construction, the \acc-marker behaves as an
indicator of definiteness.

The second fact that makes \xref{para-acc} interesting is that the marker is
optional on indefinites for some verbs (\ref{para-indef} \versus\
\ref{para-acc}), exemplifying a case of Differential Object Marking
\cttxp{aissen03}. This optionality has certain interpretative effects
that will be discussed in detail below.

The following research questions follow from the observations above:

\ex.\label{res-ques}
\a.\label{res-ques-syn} What governs the distribution of the \acc-marker? (When
is it required, when is it optional?)
\b.\label{res-ques-sem} What is the contribution of the marker in cases where it is optional?
\b.\label{res-ques-genposs} Why the marker is not optional for genitive possessive indefinites?


The aim of the present paper is to propose an analysis of the marker that
answers these three questions in a unified way.

Turkish indefinites are interesting in a couple of respects. First,
the literature on the notions like (in)definiteness, specificity,
``strength``, referentiality, and so on, usually concentrate on
subjects. This might be a confounding factor for semantic analysis of
these notions. In Turkish we are dealing with the mentioned properties
as attributed to objects rather than subjects.  Second, as I will be
arguing in the rest of the paper, the notion of ``strength'' that
needs to be associated with case-marked indefinites in Turkish is
slightly different from the similar notions in the literature. This, I
believe, makes the investigation of Turkish \acc-marked indefinites
worthwhile from a broader perspective. 

The central issue of the present paper is sometimes discussed under
the name of Differential Object Marking \cttxp{aissen03}. I agree with
\ctnm{lopez12} that ... Therefore, I will not use the terminology and
hierarchy idea of this literature.
% TODO: Complete Lopez.

Although syntactic issues are not my main concern, a few points
regarding the syntactic status of Acc-indefinites are in order.

I think the accusative case alternation on indefinite objects is not a
clear case of Differential Object Marking, which, following
\cite{lopez12}, I take to be proposing a direct morphology/semantics
relation between case and specificity (or some other interpretative
category) and that this relation operates on the basis of a
definiteness scale. I follow the Turkish literature in claiming that
there is no direct morphology/semantics relation in accusative case in
Turkish. Accusative is rather a reflex of syntactic organization,
which itself has an effect on the interpretation. A common denominator
of this literature is that the overt case is a reflection of syntactic
position and this position itself is responsible for the
interpretative differences between having and not having overt case. I
follow this idea here. Turkish nominal expressions do not carry overt
case when they remain in a domain projected by the verb, and along the
logic of \ctnm{diesing92} they thereby receive a sort of
non-referential reading in this domain.

Another reason to think that the presence of Acc-marking is a
syntactic requirement, observe that possessive DPs obligatorily
receive case-marking, regardless of their interpretative properties.

\ex. Polis Ali'nin bir fotoğrafı*(nı) arıyor.

Furthermore, as in the set of languages discussed by \cite{lopez12},
objects in small clause constructions and objects that control PRO
obligatorily receive overt case:

\ex.
\a. Bir öğrenci*(yi) zeki bulabilirim.
\b. Gözetmenlik yapmaya bir öğrenci*(yi) ikna edeceğim. 


\section{Notes on syntactic issues}

In this section I aim to provide a critical overview of the
descriptive aspects of the \acc- versus \zero-marked indefinites,
concentrating on their interpretative differences.

My syntactic assumptions are and will stay sketchy. I will also ignore
scrambling and other ``deviations'' from the canonical SOV order of
Turkish.

\section{Information structure}

% TODO: Return to Coppock and Wechsler.

\zero-indefinites can also appear in the non-focal part of an
utterace:

\ex.\label{exmezun}%
Mezunlar Dernegi'nde nadir de olsa bir profesor goruYORum.


\section{Epistemic specificity}

Use \trtx{bazı}{certain} for arguing against epistemic specificity
association for \acc-marking.


\ex.
\a. Bazı hatalar gördüm.
\b. Bazı hataları gördüm.
\b. *Bazı hatalar görmedim.
\b. Bazı hataları görmedim.


\section{Scope and word order}
\label{scscope}

I concur with \cite[3]{enc91} that semantics of \acc-indefinites is not
directly related to scope. Scopal properties come as a side-effect. 
Enc: specs can be low, but tend to be high, this must be explained.

The main argument of this section is that \acc can take the narrowest
scope, the effect is not directly related to scope, but definiteness
in the sense defended in the present paper is what allows for wide
scope.


As in many other languages, two types of indefinites differ in their
scopal behavior. Slightly adapting from \cttx{ozge11}:

Let us take an intermediate scope example:

\exg. Çogu dilbilimci önemli bir problem(-i) çözen her makale-yi okudu.\\
most linguist important a problem(-Acc) solve.Rel every article-Acc read.3sg\\
`Most linguists read every article that solves an important problem.'

In the \acc-marked version: we have three readings: (i) a single
problem; (ii) a possibly different problem per linguist; (ii) a
possibly different problem per article. In the \zero-marked version:
only the narrowest scope reading is available.


\ex. Ahmet üç kitabı okursa, sınıfı geçecek.

What are the scope possibilities here? Is a distributive reading
possible?



Scope and specificity are different things:

\ex.
\a. Karşı sahadaki parçalı bir topu sokarsan, oyunu kaybedersin.
\b. Parçalı bir top sokarsan, kaybedersin.
\b. Parçalı bir topu sokarsan, kaybedersin.


\ex.
\a. Bolivyalı bir şair barış ödülünü alamadı.
\b. Barış ödülünü Bolivyalı bir şair alamadı. 


Perhaps we can include intensional verbs here.

\ex.
\a. Ogretmen bir kitap listesi vermisti.
\b. Cogu ogrenci iki kitab(i) okudu.


\section{\encspec{ity}} %chungladusaw still cite enc for specificity.
\label{scpartspec}

The seminal work of \ctnm{enc91}, besides being the most influential
account of the interpretative effect of \acc-marking on Turkish
indefinites, provides some important insights for noun phrase
semantics in general. The significance of this work for our present
purposes is likewise twofold: (i) I will expose a critical explanatory
weakness of \ctnms{enc91} account and then modify it into a more
successful explanation of the facts at issue; and (ii) I will claim
that there are important lessons to draw from this modified account
for the noun phrase semantics in general.

\ctnm{enc91} claims that a Turkish indefinite noun phrase carries the
\acc-marker if and only if it is ``specific''.  \ctnm{enc91} relates
her notion of ``specificity'' to ``partitivity'', ``strong''/``weak''
\cttxp{milsark77} and to D(iscourse)-linking \cttxp{pesetsky87}. Care
should be taken to distinguish this notion of ``specificity'' from
other varieties of this notoriously elusive category (see
\cttx{farkas02a,heusinger11} for guidance into the literature). For this
reason, I will call the notion of ``specificity'' advocated by
\ctnm{enc91} \encspec{ity}.

Here is her famous example illustrating \encspec{ity}.

\exg.\label{exencintro}%
{Odam-a}  {birka\c{c}} {\c{c}ocuk} {girdi.}\\
{my-room-Dat} {several}  {child}  {entered}\\
`Several children entered my room.'

\ex.\label{exenc}
\ag.\label{exenca}İki  kız-\textbf{ı} tanıyordum.\\
{two}  {girl-{\bf Acc}}  {knew.1sg}\\
`I knew two girls (among the children). '\hfill (\encspec)
\bg.\label{exencz}\#İki  kız  tanıyordum. \\
{two}  {girl}  {knew.1sg}\\
`I knew two girls.'\hfill (non-\encspec)

\ctnm{enc91} observes that in a discourse initiated by
\xref{exencintro}, only with the form \xref{exenca}, where the
indefinite \trtx{iki kız}{`two girls'} is \acc-marked, the girls are
understood to be coming from the set of children recently introduced;
whereas in the absence of the marker as in \xref{exencz}, the girls
are indicated to be independent from the set of children,
resulting in pragmatic anomaly. 


Before I embark on a critical assessment of this important proposal, I
would like to cast it in a formal framework in order both to clarify
\ctnms{enc91} proposal and to lay a foundation on which I will build
my modifications. My choice of framework is Discourse Representation
Theory \cttxp{kamp81,kampreyle93}, equipped with the Binding Theory
of Presupposition Justification \cttxp{sandt92,geurts99}. I will have
more to say below on this particular choice of framework.

\citeauthor{enc91} starts with a standard dynamic model of noun phrase
semantics, which associates every noun phrase with an index (or a
discourse referent) that gets bound by an operator external to the NP
itself. \citeauthor{enc91} extends this picture by adding a second
index, standing for a \textbf{superset} of the first index.  Let us
call the latter the superset index and the former referent index.
Here is the formal definition from \cttx{enc91}:

\ex.\label{encform} Every $[_{\text{NP}}\  \alpha ]_{\langle i,j\rangle}$ is interpreted as
$\alpha(x_i)$ and\\
$x_i \subseteq x_j$ if $\text{NP}_{\langle i,j\rangle}$ is plural;\\
$\{x_i\} \subseteq x_j$ if $\text{NP}_{\langle i,j\rangle}$ is
singular.

\ctnm{enc91} further claims that the usual definiteness feature
([+definite] for familiar, [-definite] for novel) applies separately
to both the referent and the superset index. In this setting, a
standard definite NP has [+definite], and a standard indefinite has
[-definite] on their both indices.  \encspec{ity} corresponds to the
case where the referent index has [-definite] while the superset index
has [+definite].

Let us see how the proposal works, over \xref{exenc}. The discourse
opener \xref{exencintro} contributes the simplified main DRS in
\xref{exencmaindrs}, containing the referents for the speaker and a
set of children:\footnote{I follow the tradition of distinguishing
constants from variables by ``priming'' the former.}

\ex.\label{exencmaindrs}
\a. Several children entered my room.
\b. \udrs{s',z}{children'(z)}

First take the non-case-marked continuation:

\ex.\label{exenczerodrs}
\a. I knew two girls.\hfill (non-case-marked)
\b.
\udrs{s',z,x_1,x_2}{children'(z)\\
two\mbox{-}girls'(x_1) \quad x_1 \subseteq x_2 \quad know'(s',x_1)
}

Indices $x_1$ and $x_2$ are both indefinite and therefore carry a
novelty condition. We assume that this novelty condition makes sure
that the set type indices $x_1$ and $x_2$ are disjoint with any
already established index, a condition not included in
(\ref{exenczerodrs}b).  Thanks to the novelty condition on both
indices, the girls are understood to be not included in the set of
children established in the discourse model, eventually giving rise to
inappropriateness in the context of \xref{exencintro}, which is
expected as the introduction of the set of children is left
unmotivated and therefore does not cohere with what comes after it.

Let us now turn to the analysis of the case-marked continuation
\xref{exenca}.

\ex.\label{exencacc1}
\a. I knew two girls.\hfill (Acc-marked)
\b. \udrs{s',z,x_1,\anaph{x_2}}{children'(z)\quad  entered'(z)\\
two\mbox{-}girls'(x_1) \quad x_1 \subseteq \anaph{x_2} \quad know'(s',x_1)
}

Here the direct object \trtx{iki kız-ı}{two girl-Acc} again contributes two
indices, but this time the second index standing for a superset is definite. In
modelling definiteness of this sort, I follow the binding theory of
presupposition resolution of \cttx{sandt92}. I think at this stage nothing
crucial hinges on the choice of definiteness account and its formal
representation. In this theory, presupposition resolution is treated on a par
with anaphora resolution. Underlining a discourse referent, as is done in
\xref{exencacc1}, indicates that the underlined content needs an antecedent to
get bound. This binding relation between the antecedent and the
presuppositional content obeys the usual accessibility constraints of DRT.
Therefore, for the interpretation in \xref{exencacc1} to converge this binding
requirement needs to be satisfied. Presupposition resolution in this fashion is
a non-deterministic process that obeys certain restrictions
\cttxp{geurts10}:
% TODO: elaboration and further citations

\ex.
\a. The antecedent should be accessible.
\b. If there is an antecedent accessible in a near distance bind to
that; unless there are inferential restrictions.
\b. If an antecedent is not present, accommodate one -- this
accommodation process is guided again by inference and global
is favored over local accommodation.

In the present case the presupposition carried by the superset index $x_2$ is
resolved through identification with the set $z$ of children already
established in the main DRS:

\ex.
\udrs{s',z,x_1,\anaph{x_2}}{children'(z)\\
two\mbox{-}girls'(x_1) \quad x_1 \subseteq \anaph{x_2} \quad know'(s',x_1)\\
\anaph{x_2} = z
}


If all the presuppositions of a DRS are bound, than the algebraic
cancellation of identical terms yields a DRS without any underlined
terms. This holds for the current case. The representation we arrive
at correctly captures Enç's observation for the \acc-marked
continuation \xref{exenca}:

\ex.\udrs{s',z,x_1}{children'(z)\\
two\mbox{-}girls'(x_1) \quad x_1 \subseteq z \quad know'(s',x_1)
}

\subsection{Issues with \encspec{ity}}

\ctnms{enc91} proposal has been empirically challenged on the basis of
both \acc-marked indefinites that are \emph{not} \encspec\ and
\zero-marked indefinites that \emph{are} \encspec\
(\cttx{taylanzimmer94, zidani97, kelepir01, heusingerkornfilt05,
heusingerkornfilt17, kilicaslan06, issever07, nakipoglu09, ozge11}
among others).  In this paper, I will gloss over such gaps in the data
regarding the semantic effects of the \acc-marker and concentrate on
the cases where the presence versus absence of the marker has a clear
effect as in \ctnms{enc91} example \xref{exenc}.  I will show
in Section~\ref{sccontraenc} that \encspec{ity}, as the semantic
correlate of \acc-marking, cannot hold up to the facts of Turkish,
even when we concentrate on such clear cases. Before that, in the next
section, we need to take guard against a potential misunderstanding
concerning the semantics of \acc-marked indefinites. 

%I will instead argue that the semantics of the marker is more closely
%related to ``strong''/``weak'' distinction of \ctnm{milsark77}.  This
%might appear paradoxical, given that \ctnm{enc91} equates her notion
%of ``specificity'' with ``strong''/``weak'' (and also with
%D(iscourse)-linking of \ctnm{pesetsky87}), I will show, however, that
%this equivalence does not hold.  \encspec[ity} is implicit domain
%restriction, which is a weaker relation than ``strength''; and it is
%not the semantic property that the Turkish \acc-marker indicates. In
%order to proceed in this direction, first we need to look at \encspec
%ity in more detail.

\subsubsection{Covert partitivity}
\label{sccovertpartitivity}

\ctnm{enc91} analyzes the semantics of specificity in two classes.
In the first class, ``link by inclusion'', there is a superset
antecedent established in the context for the indefinite; specificity
is equivalent to covert partitivity in this case. The second class,
``link by assignment'', includes the cases where referent of the
indefinite is linked to its antecedent by a designated function
recoverable from the context ala \cite{hintikka86}, in the presence of
specificity adjectives like \emph{certain} (or Turkish
\trtx{belli}{certain}) and/or verbs of assignment like \emph{give},
\emph{appoint}, \emph{assign}.\footnote{\cite{enc91} slightly extends
\cite{hintikka86} in ways that do not concern us here.}

\cite{enc91} explicitly associates her notion of specificity with
partitivity, taking \xref{exenca} to be equivalent to the
following overt partitive (her Ex.\ 19):

\ex.\label{exovertpart} Kız-lar-dan iki-sin-i tanıyordum.\\
girl-Plu-Abl two-Agr-Acc I-knew\\
`I knew two of the girls.'

This observation, however, is not correct: \xref{exenca} is not
equivalent to \xref{exovertpart}, because the latter, but not the
former, implies that there are more than two girls in the context, a
typical implication of overt partitives. For the \acc-marked
indefinite in \ref{exenca}, there may be just two or more girls among
the children who entered the room, \acc-marked indefinite has simply
no commitment on this count. For this reason when you make the
explicit partitive a continuation to the discourse opener, what you
get is anomaly. The reason is that the presupposition triggered by the
explicit partitive is hard to accommodate in the present context.

One case that foregrounds a partitive interpretation is when the
antecedent noun and the head of the indefinite are the
same:\footnote{Besides changing the head noun of the \acc-marked
indefinite, I also adapted the opening sentence
\xref{exencsamehead}{a}, because the modifier
\trtx{birkaç}{several/a few} in the original already implies that
there were more than two children entering the room.}

\ex.\label{exencsamehead}
\ag.\label{exencsameheada}{Odam-a}  {\c{c}ocuk-lar} {girdi.}\\
{my-room-Dat} {child-Plu}  {entered}\\
`(Some) children entered my room.'
\bg.\label{exencsameheadb}İki  çocuğ-\textbf{u} tanıyordum.\\
{two}  {child-{\bf Acc}}  {knew.1sg}\\
`I knew two children (among the children). '\hfill (\encspec)


\xref{exencsameheadb} do imply that there are more than two
children in the context.\footnote{Note that \xref{exencsameheadb}
has also a definite reading we are not interested in here.} This
implication, however, can be derived via the Gricean inference ``If
the number of children entering the speaker's room were two, then the
speaker would not use the weaker determiner \trtx{bir kaç}{several/a
few} in \xref{exencsameheada}'', and therefore need not be coded in
the semantics of the \acc-marked indefinite.

A second option for an \encspec\ indefinite is to be linked to an
already established discourse referent with an assignment or some
other recoverable relation. This also does not seem to hold for the
present example. Any head noun that do not conceptually clash with
\emph{child} would be appropriate in place of \emph{girl}.

Given these facts, I conclude that the following hold. 

\ex.
\a. Although \acc-marked indefinites can yield partitive
interpretations, this  does not follow from the semantics of the NP,
but rather from a property of the context.
\b. The linking relation does not have to be regulated by a systematic
assignment, it can just be circumstantial.


%and/or ``strong'' interpretation in the sense of \cite{milsark77} and
%especially \cite{diesing92}. 

%
%Acc-marking is not strong/weak in the sense of Diesing (1992). In sm
%apples versus some apples, the strong version implies partitivity.
%
%The same difference goes for Diesingian ``strong'' indefinites. Take
%an example from \cttx[ex.\ 9]{vonfintel98}, slightly shortened:
%
%\ex.\label{exfintelstrong}I’m not sure yet whether there any mistakes at all in this book manuscript, but
%we can definitely not publish it
%\a. If some major mistakes are found.
%\b. \label{exfintelstrongb}\#If some mistakes are major.
%
%
%The ``strong'' (or presuppositional) indefinite in
%\xref{exfintelstrongb} is substitutable with \emph{some of the
%mistakes}. Turkish \acc-marked indefinites diverge from this pattern
%in not implying partitivity. 
%
%On the reverse side, Turkish \acc-indefinites do not imply
%exhaustivity either. 

\subsubsection{Implicit domain restriction} \label{sccontraenc}
\label{scencneg}

In this section, I will show that the semantics formulated by
\ctnm{enc91} is implicit domain restriction,\footnote{\cttx{nilsson85}
is the first author to attribute implicit domain restriction to
\acc-marking of Turkish indefinites, as far as I know.} This makes
\encspec{ity} an unsatisfactory account of Turkish \acc-indefinites,
since \acc-marking indicates a stronger relation than implicit domain
restriction.

Having seen how \encspec{ity}works for Enç's example \xref{exenc}, let
us modify the example to be able to derive further predictions of her
account. Here is a minimal pair, where the verb \trtx{tanı}{know} is
negated:

\exg.\label{exenc16neg}%
{Odam-a}  {birka\c{c}} {\c{c}ocuk} {girdi.}\\
{my-room-Dat} {several}  {child}  {entered}\\
`Several children entered my room.'

\ex.\label{exenc17neg}
\ag.\label{exencnegacc}{\.Iki}  {k\i{}z-\textbf{\i}} {tan\i{}-m-ıyordum}.\\
{two}  {girl-{\bf Acc}}  {knew.Neg.1sg}\\
`I didn't know two girls (among the children). '\hfill (case-marked)
\bg.\label{exencnegzero}\#{\.Iki}  {k\i{}z}  {tan\i{}-m-ıyordum.} \\
{two}  {girl}  {knew.Neg.1sg}\\
`I didn't know two girls.'\hfill (non-case-marked)

Let us analyze the interpretations assigned to these examples by
\ctnms{enc91} account. I will initially assume that negation in
\xref{exenc16neg} is at VP level, which contains the verb and the
direct object in both case-marked and non-case-marked variants. Under
this assumption, the non-case marked \xref{exencnegzero} gets the
following interpretation:

\ex.\label{exencnegzerodrs}
\udrs{s',z}{children'(z)\\
	\udrs[\neg]{x_1,x_2}{
	two\mbox{-}girls'(x_1) \quad x_1 \subseteq x_2  \quad know'(s',x_1)
	}
}

Given that $x_2$ is [-definite] and therefore disjoint with $z$,
\xref{exencnegzerodrs} is satisfied in a model where it is impossible
to find at least two girls known to the speaker that are outside the
set of children.   As in the non-negated original example
\xref{exencz}, \xref{exencnegzero} hardly coheres with
\xref{exenc16neg}, making it difficult to obtain reliable judgments.
For this reason, although the interpretation appears to be
unsatisfactory,\footnote{\xref{exencnegzerodrs} would be satisfied in
a model where there are girls known to the speaker among the children,
which appears to confuse the native speaker judgments.} I leave it as
an open empirical question whether \xref{exencnegzerodrs} is the
correct interpretation or not, as this is not central to our present
concerns.

Let us now turn to the case-marked variant \xref{exencnegacc},
which is interpreted in the following three steps:

\ex.
\a.
\udrs{s',z}{children'(z)\\
	\udrs[\neg]{x_1,\anaph{x_2}}{
	two\mbox{-}girls'(x_1) \quad x_1 \subseteq \anaph{x_2}  \quad know'(s',x_1)
	}
}
\b.
\udrs{s',z}{children'(z)\\
	\udrs[\neg]{x_1,\anaph{x_2}}{
	two\mbox{-}girls'(x_1) \quad x_1 \subseteq \anaph{x_2}  \quad know'(s',x_1)\\
	\anaph{x_2} = z
	}
}
\b.\label{exencnegaccdrs}
\udrs{s',z}{children'(z)\\
	\udrs[\neg]{x_1}{
	two\mbox{-}girls'(x_1) \quad x_1 \subseteq z  \quad know'(s',x_1)
	}
}

The DRS \xref{exencnegaccdrs} is satisfied in any model where it is
impossible to find at least two girls known to the speaker within the
set of children. A critical observation here is that
\xref{exencnegaccdrs} gets satisfied in a model where there were no
girls \label{at} all among the children. By this token,
\xref{exencnegaccdrs} diverges from the actual meaning of
\xref{exencnegacc}: In its primary reading, \xref{exencnegacc} gets
satisfied only if there are at least two girls among the children
unknown to the speaker, which is represented in DRT as follows:\footnote{The reading
		\xref{exencwide} is not the only reading \xref{exencnegacc}
		can get. I will discuss another type of reading for such
		sentences below. What is crucial for now is that
		\xref{exencnegacc} simply lacks any reading that does
\emph{not} commit to the existence of at least two girls among the
children.}


\ex.\label{exencwide}
\udrs{s',z,x_1}{children'(z)\\
	two\mbox{-}girls'(x_1) \quad x_1 \subseteq z\\
	\udrs[\neg]{}{
	know'(s',x_1)
	}
}

Therefore, \ctnms{enc91} semantic formulation assigns the wrong
interpretation to \xref{exencnegacc}.


Is there a way to keep \encspec{ity} as it is and resort to other factors to
explain the incorrect predictions of the account? \ctnm{enc91} proposes such a
solution.\footnote{From this proposal, I infer that she is already
aware of the problem, although she does not discuss negation.} One
major aim of \ctnm{enc91} is to motivate a notion of specificity that
is orthogonal to scope phenomena. It is well-known that the two
notions are closely related \cttxp{farkas02a}. \ctnm{enc91} first observes that
\acc-marked indefinites tend to take wide-scope. She explains this
fact by claiming that in cases where case-marking and non-marking
yields the same interpretation, the case-marked version is interpreted
as taking wider-scope with respect to a commanding operator, through a
Gricean inference.

Adapting the argument to the present case would yield,

\ex.\label{encgrice}
\a. \xref{exencnegaccdrs} is the semantic interpretation that
would be assigned by the grammar to \xref{exencnegacc};
\b. The non-case-marked version is assigned the same interpretation by
the grammar. Therefore,  \xref{exencnegaccdrs} and
\xref{exencnegzerodrs} are equivalent.
\b. Because of this equivalence, the hearer of \xref{exencnegacc}
unconsciously reasons as follows: ``The speaker could have expressed
the same content with a non-marked version, but she uses the marked
one. Therefore, she is trying to convey a non-standard interpretation,
which I take to be the one where the indefinite takes scope over
negation''. As a result of this reasoning, the hearer ends up with
interpreting \xref{exencnegacc} as \xref{exencwide}.


An immediate problem with this argument concerns the equivalence of
\xref{exencnegaccdrs} and \xref{exencnegzerodrs}. This cannot be
truth-conditional equivalence, since in a model where the speaker
knows two girls outside of the children set, the former representation
gets satisfied while the latter does not. Therefore, the argument in
\xref{encgrice} is at least in need of specifying a notion of
equivalence that would hold between these two representations, such
that it will serve the ground for the Gricean reasoning proposed.
Otherwise, the reason why \xref{exencnegacc} is not understood in the
way predicted by \ctnms{enc91} account remains unexplained.

There is another potential source of the wide-scope behavior of the
indefinite in \xref{exencnegacc}, namely the syntax. The indefinite
could be forced to move out of its local domain for case-checking or
some other reason along the lines of \ctnms{diesing92} Mapping
Hypothesis or some variant of it, resulting in a structure
like:\footnote{For the syntactic position of \acc-marked indefinites
see \cttx{kelepir01,aydemir04,ozturk05}.}

\ex. [\ldots two girls \ldots [$_{NegP}$ \ldots \sout{two girls} \ldots ] \ldots]


With this syntactically motivated movement, one can formulate
the following:

\ex.\label{wsa}{\it Forced wide-scope account of \acc-marked
indefinites:}
\a.\label{wsaa}The accusative case on Turkish indefinite direct objects marks
\encspec{ity}, as formulated in \xref{encform}. 
\b.\label{wsab}  An accusative marked direct object is required to raise to a
position that is at least higher than the phrasal
negation.\footnote{Turkish also has a clausal negation operator 
 \textit{değil}.}


If it can be maintained, \xref{wsa} can explain why
\ctnms{enc91} prediction for the interpretation of \xref{exencnegacc}
is not fulfilled; it is \xref{wsaa}  \emph{and}
\xref{wsab} working together that causes
\xref{exencnegacc} get interpreted as \xref{exencwide} rather than as
\xref{exencnegaccdrs}.

I argue, however, that \xref{wsa} can at best be a partial account
of the semantics of \acc-marked indefinites in Turkish. The reason is
that \acc-marked indefinites do not necessarily get a wide-scope
reading as was the case for \xref{exencnegacc}. My argument
requires a closer look at the scopal semantics of \acc-marked
indefinites, to which I will directly turn. 

Let me diverge from \ctnms{enc91} example, to return to it below, and
replace the numeral \trtx{iki}{`two'} with the indefinite determiner
\trtx{bir}{`a/one'} and alter the syntactic environment from negation
to antecedent of a conditional:\footnote{The example is inspired by
\cttx{vonfintel98}.}

% TODO: Basit hatalari gormesi onemli degil; yeter ki kritik bir hatayi gormesin.

\ex.\label{excond}%
Context: John is a referee reviewing an article under discussion.
\ag.\label{exconda}%
	John (kritik) bir hata{\bf-yı} görür-se, makale-yi reddeder.\\
	J. critical a error{\bf-Acc} sees-Cond article-Acc rejects\\
	Rd.\ 1: `If John sees an (critical) error, he'll reject the article.'' (It's
	common ground that there are (critical) errors in the article.)\\
	Rd.\ 2: `An (critical) error is such that if John sees it, he'll reject the
	article.
\bg.\label{excondz}%
	John (kritik) bir hata görür-se, makale-yi reddeder.\\
	J. critical a error sees-Cond article-Acc rejects\\
	`If John sees an (critical) error, he'll reject the article.'' (no commitment
	to the existence of errors.)


A crucial difference between \xref{exconda} and \xref{excondz} is that
while the former presupposes the existence of errors in the article,
no such presupposition is involved in the latter. Another important
fact concerning \xref{exconda} is that besides a wide-scope reading of
the indefinite, it has another reading where the indefinite stays
within the scope of the antecedent of the conditional, while its
restrictor stays out. I will call this latter type of reading ``hybrid
scope'' reading. The two readings are rendered as follows in
DRT:\footnote{\ref{excondanar} is an instance of covert partitivity in
\ctnms{enc91} terminology.}

\ex. 
\a.\label{excondawide}Wide scope reading:
\udrs{j',x,y}{article'(x) \quad error'(y)\\
\ccon
{\udrs{}{see'(j',y)}}
{\udrs{}{reject'(j',x)}}
{\Rightarrow}
}
\b.\label{excondanar}Hybrid scope reading:
\udrs{j',x,Y}{article'(x) \quad error'(Y)\\
\ccon
{\udrs{z}{z\in Y \quad see'(j',z)}}
{\udrs{}{reject'(j',x)}}
{\Rightarrow}
}


The non-case marked variant \xref{excondz} is interpreted as follows:

\ex. \udrs{j',x}{article'(x)\\
\ccon
{\udrs{y}{error'(y) \quad see'(j',y)}}
{\udrs{}{reject'(j',x)}}
{\Rightarrow}
}


The hybrid scope reading \xref{excondanar}, which is the primary reading
for \xref{exconda} in a context where it is common ground that there are
a multiplicity of errors all equally critical, is impossible to arrive
at by coupling \encspec{}ity and wide-scope behavior, namely the
account formulated in \xref{wsa}. What is predicted by
that account instead is only \xref{excondawide}. For this reason,
\xref{wsa} fails to be adequate for the semantics of
\acc-marked indefinites in Turkish.

\ctnm{enc91} also has a hybrid scope example involving the intensional verb
\trtx{iste}{want}:

\ex.Context: A musical instrument store.
\ag. Ali bir piyano-yu kirala-mak istiyor.\\
A. a piano-Acc hire-INF wants\\
Rd.\ 1: `Ali wants to hire a piano from among those in the store (but he did not decide which).'\\
Rd.\ 2: `There is a piano from among those in the store such that Ali wants to hire it.'

Again, what I translate as ``Rd.\ 1'' is not possible to get via
\xref{wsa}.\footnote{A potential objection can be
		based on the fact that \xref{wsa} is formulated on
		the basis of scope in relation to negation, not intensional
verbs. This objection is not viable, because hybrid scope is
still available with negation on the embedded or the matrix verb.}

The scopal behavior of \acc-marked indefinites exemplified above is
quite general. It is straightforward to have similar examples with
other types of scope variation inducing operators. Here is one with
the imperative, assume a context of two interviewers discussing what
to ask to an interviewee:

\ex.
\ag.\label{impz}{On-a} {zor} {bir} {soru} {sor}.\\
{her-Dat} {hard} {a} {question} {ask}\\
`Ask her a difficult question.'
\bg.\label{impa}{On-a} {zor} {bir} {soru\bf{-yu}} {sor}.\\
{her-Dat} {hard} {a} {question\bf{-Acc}} {ask}\\
`Ask her a hard question.'\hfill (D-linked)\\
Rd.\ 1: `Ask her one of the difficult questions.'\\
Rd.\ 2: ??A specific question is intended, to be continued by naming the question
explicitly.

This time, the hybrid scope reading appears to be the only
pragmatically plausible interpretation of \xref{impa}.

% TODO: herhangi ve digerleri..

Let me return to negation. In order to avoid the interference of
meta-linguistic negation, I follow \ctnms{szabolcsi04} suggestion of
using reason contextualization. Take an exam context where the
a test-taker looks happy after the exam. When inquired about the source of
her happiness, both answers below are appropriate:  

\ex.
\a.\label{sorunegz}Öğretmen zor bir soru sor-ma-dı.\\
teacher difficult a question ask-Neg-Past\\
`The teacher didn't ask a difficult question.' (No commitment to the
existence of difficult questions beforehand.)
\b.\label{sorunega}Öğretmen zor bir soru{\bf-yu} sor-ma-dı.\\
teacher difficult a question{\bf-Acc} ask-Neg-Past\\
Rd.\ 1: `The teacher didn't ask a difficult question.' (It's
common-ground that there were difficult questions prepared beforehand.)\\
Rd.\ 2: `There was a question that the teacher (fortunately) didn't ask.'

Again the crucial observation is that \xref{sorunega} necessarily
commits to the existence of difficult questions known to the examinee. No
such existence presupposition is present in the non-marked variant
\xref{sorunegz}.\footnote{A more straightforward way of
		expressing the hybrid scope reading might be to use an overt
		partitive like:

		\ex. Öğretmen zor sorulardan birini sor-ma-dı.\\
		`The teacher didn't ask one of the difficult questions.'

		However, as I argued above, equivalence to an overt partitive
		applies only in certain contexts. The point here, on the other
		hand, is that the \acc-indefinite \emph{can} be used to convey
		the meaning that can be conveyed by an overt partitive.  There
		will be cases below where \acc-marked indefinite is the only
		option to convey a hybrid scope reading.

		Also note that the hybrid scope reading with an \acc-marked
		indefinite is more readily available with a clausal model like
		\emph{fortunately}: 

		\exg. Çok şükür öğretmen zor bir soru{\bf-yu} sor-ma-dı.\\
		much gratitude teacher difficult a question{\bf-Acc} ask-Neg-Past\\
		`Fortunately/Thank God The teacher didn't ask a difficult question'
		(It is common-ground that there were difficult questions prepared
		beforehand.)

		This is expected, since, as \cttx{krifka08} observes,
		adverbials like \emph{fortunately} facilitate the comparison of
		Roothian alternatives, in this case putting the comparison of
		difficult versus easy questions in perspective.}

In all the examples we had so far, the restrictor of the indefinite
and the contextually established set that the indefinite is linked to
are identical. In \ctnms{enc91} example \xref{exenc}, however, the
restrictor of the indefinite (\intx{girls}) and the antecedent set
(\intx{children}) were different. While discussing this example, I
underlined the fact that the effect of the marker was not simply a
partitive reading like ``two of the girls.'' Here I provide a model
example where this is again the case and we have a hybrid scope
reading, as in other examples I discussed above. My aim is again to
show that \xref{wsa} cannot deliver all the available
readings.
 
Take a scenario where Alice has a number of dogs in her farm and her
niece Betty comes to visit her. Betty wants a dog among them as a
birthday present and she wants a Retriever. Alice gives the present,
but Betty does not look happy with what she got. Someone asks the
reason for Betty's not being happy.  Alice answers:\footnote{The verb
		\trtx{hediye et}{present make} is a light verb construction
		that behaves identically to a lexical verb, as far as the
grammar of zero versus case-marked objects are concerned.  Similar examples
can be constructed with lexical verbs as well.}

\ex.
\ag.\label{exretz}Çünkü on-a bir Retriever hediye et-me-dim.\\
 because her-Dat a Retriever present make-Neg-Past.1sg\\
`Because I didn't give her a Retriever as present.'
\bg.\label{exreta}Çünkü on-a bir Retriever-ı hediye et-me-dim.\\ 
because her-Dat a Retriever-Acc present make-Neg-Past.1sg\\
`Because I didn't give her a Retriever as present.'


As expected \zero-marked \xref{exretz} does not link to the previous
discourse and therefore is fine out-of-the-blue. The \acc-marked
version \xref{exreta} on the other hand is inappropriate in the
absence of the antecedent set of dogs;\footnote{Any set that comprises
of items to be given as present would also provide the appropriate
context.} and it is ambiguous between a wide scope and a hybrid scope
reading. Once again \ctnm{enc91} semantics coupled with forced wide
scope for \acc-indefinites can only deliver the wide scope reading and
therefore is not adequate in capturing all the available readings.

Finally, one might ask why a hybrid scope reading is not available for
the negation of \ctnms{enc91} example given as \xref{exencnegacc}
above.  Furthermore, one might think that the absence of that sort of
reading for \xref{exencnegacc} is due to having the numeral
\trtx{iki}{two} instead of \trtx{bir}{a/one}. The reason is that the
typical context accommodated for the example cannot support that sort
of reading, namely it is hard to provide a potential reason for the
speaker to deny that she knows two girls from among the children
entering the room. Once the needed contextual support is provided,
hybrid scope readings arise for the numeral \trtx{iki}{two} as well.

For instance, imagine that an even number of children are to be
assigned to dormitory rooms, each room accommodating two. Also imagine
that there is a rule dictating not to put two girls in the same room,
rooms should be either two boys or a boy and a girl. The speaker of
\xref{odaa} is successful in stating that she observed the rule
applied to the bunch of children already established in the discourse,
given our scenario. In the absence of such an established set of
children, the appropriate form would be \xref{odaz} and the
\acc-marked version \xref{odaa} would be inappropriate.\footnote{I am
		grateful to Daniel Büring for pointing these types of contexts
out to me.} 

\ex.
\ag.\label{odaz}Aynı oda-ya iki kız koy-ma-dım.\\
same room-Dat  two girl put-Neg-Past.1sg\\
``I didn't put two girls in the same room.'
\bg.\label{odaa}Aynı oda-ya iki kız-ı koy-ma-dım.\\
same room-Dat  two girl-Acc put-Neg-Past.1sg\\
``I didn't put two girls in the same room.'

To sum up, in this section we first saw that \ctnm{enc91} semantics
for \acc-marked indefinites is inadequate on its own to account for
the fact that the use of an \acc-indefinite commits the user to the
existence of the restrictor of the indefinite (``existential
import''). Then, we saw that a syntactically motivated wide-scope
constraint for an \acc-indefinite cannot remedy this deficiency. The
reason was the existence of cases of hybrid scope in which while the
restrictor is wide-scope, the discourse referent associated with the
indefinite stays narrow. Now I turn to a proposal that aims to provide
a more satisfactory semantics for \acc-marked indefinites in Turkish.


% TODO: This is of course different than implicit domain restriction,
% as it involves some familiarity for the way the domain is
% restricted.


% TODO: Can I make acc-effect came by the mechanics of DRT. When the
% NP is outside of the root DRS it gets interpreted strongly, case is
% just a syntactic correlate of this outsideness. But then what to do
% with scrambled acc's: maybe they are inside the VP anyway
% (gracanin-issever).

% TODO: Ahmet giderken kitaplarini bize birakmisti. Umut gecen
% haftasonu bir romani bitirdi. Asiye de bu haftasonu/ Asiye de oyle
%
% Makaleyi ikimiz de okuduk. Ben kritik bir hatayi gormemisim. Asiye
% de oyle.
%
% Makaleyi ikimiz de okuduk. Ben kritik iki hatayi gormemisim. Asiye
% de uc.
% 
% Mehmet de sen de zor bir soru(yu) sorun. Zor bir soru sor/hazirla.
% Zor bir soruyu sor/*hazria.
% 
% Mehmet (nispeten) zor bir soruyu sordu. Ahmet de oyle.


% TODO: Binding heuristics: prefer binding over accom; prefer global
% over local.

\section{The proposal}
\label{scproposal}

% TODO: Identifiability vs. familiarity.

As the data we considered above show, \encspec{ity} misses the crucial
property of ``existential import'' associated with \acc-marked
indefinites in Turkish.  In relation to an operator,
\acc-indefinites can receive a wide scope reading and a hybrid scope
reading. In the latter case, the restrictor takes wide scope while the
discourse marker stays narrow. It is almost never the case that both
restrictor and the marker falls within the scope of a commanding
operator at the same time.\footnote{See below for an exception.} 

These facts suggest that the central aspect of \acc-marking for
indefinites is ``strength''. In an attempt to clarify the grammar of
existential sentences in English, \cite{milsark77} introduced the
``strong''/``weak'' distinction for determiners. ``Strong''
determiners are those that quantify over a domain denoted by their
restrictor terms. A typical effect of ``strength'' is that the
restrictor predicate is presupposed to exist at the point of
quantification. One can think of this as a two step process: first you
fix the domain, then you go on with quantifying over that domain.
``Weak'' determiners on the other hand lack quantificational force of
their own; what they provide is a cardinality predicate that specifies
the size of the restrictor.  ``Strongly'' determined NPs overlap with
the syntactic/semantic notion of definiteness but not completely.
\ctnm{milsark77} credits \ctnm{postal66} for the observation that
``strong''/``weak'' cross cuts the territory of indefiniteness.  Being
``strong'' implies presuppositionality by definition, where what is
presupposed is existence. My proposal will be built over the idea
that \acc-marking indicates are ``strong'', on the basis of
the observation made above that \acc-indefinites project existence
presuppositions at standard test environments like antecedent of a
conditional, imperatives and negation.\footnote{The present work is
		not the first to argue that \acc-indefinites are
		presuppositional. Remember that \cttx{enc91} also associates
		her notion of specificity with ``strength'' in the sense I am
		using the term here. We saw above that her semantics falls
		short of modeling this notion.  \cttx{ketrez05} and
		\cttx{kelepir01} also attribute presuppositionality to \acc-marking.
		As far as I know, the present paper is, however, the first to make a
		detailed case for presuppositionality and model its semantics.}



\subsection{Partitive/non-partitive and determiner/numeral}

Before I state my proposal, I would like to analyze and clarify two
binary distinctions. One is the distinction between having and not
having a partitive interpretation. What I mean by partitive
interpretation is the case where the use of an expression \emph{bir
$p$} implies that there are more $p$s in the context. The second
distinction is between a numeral and a determiner reading for the
expression \emph{bir}. I distinguish these readings with the aid of
English translations: the expression \emph{bir} results in a numeral
reading when it gets translated as \emph{one}, and results in a
determiner reading when it gets translated  as \emph{a(n)}. 

Let us organize the discussion around two cases: 1.\ the restrictors
of the indefinite and the antecedent are identical, 2.\ they are not
identical. Initially, the identity I mean here is surface (or string)
identity, not set-theoretic (or denotational) identity. I exemplify
the distinction over an example where I manipulate the restrictor of
the antecedent.  The alternative opening sentences are given in
\xref{exromanintro} and the alternative case-markings are given in
\xref{exromancont}.\footnote{When I test this type of examples with
		native speakers in their from without a modifier
		(\trtx{kalın}{thick} in this case), the response I usually get
		is: Well, this is grammatical, but a more natural expression
		would be:
		
\ex. Geçen Pazar (romanlardan) birini okudum.

}

\ex.\label{exromanintro}
\a.\label{exromanirom}Ali bana bir koli roman bıraktı.\hfill{(identical restrictors)}
\b.\label{exromanikit}Ali bana bir koli kitap bıraktı.\hfill{(non-identical restrictors)}

\ex.\label{exromancont}
\a.\label{exromanz}Geçen Pazar (kalın) bir roman okudum.\hfill{(\zero-marked continuation)}
\b.\label{exromana}Geçen Pazar (kalın) bir romanı
okudum.\hfill{(\acc-marked continuation)}


We first put \xref{exromanz} out of our way. The \zero-marked version
is not an appropriate continuation to neither \xref{exromanikit} nor
\xref{exromanirom}.

The \acc-marked \xref{exromana} implies partitivity (=there is more
than one novel in the box) when ıt follows \xref{exromanirom}, but
loses this implication when uttered after \xref{exromanikit}. There is
no reason to expect this difference stem from the \acc-marked
indefinite itself; a satisfactory account should make it follow from
the change in the context.

I claim that \xref{exromana} results in a partitive reading through
inference. The expression \emph{a box of novels} entails a
multiplicity of novels and the indefinite specifies the number of read
novels as one, therefore, it is straightforward to get the
interpretation ``one of the novels.'' Furthermore, this inference is
independent of the number specification in the indefinite. In the case
of the speaker reading two novels last night, since a box of novels
normally contains more than two novels, hearers are still expected to
get the ``two of the novels'' reading. Likewise for greater numerals
as long as the number stays within pragmatic limits.

An observation that is more crucial for my proposal is the following.
When restrictors overlap, Roothian (\citeyear{rooth92}) alternatives
are activated along the dimension of cardinality rather than the
restrictor, and the numeral gets intonationally focused.\footnote{The
focus pattern of \emph{bir} is observed by \cttx{kornfilt97}.} The
most proper English translation would be ``one novel''.

When the context provides the antecedent under a description different
from that of the indefinite, we lose the partitive interpretation.
The same example \xref{exromana} posed as a continuation to
\xref{exromanikit} does not imply that there was more than one novel
in the box, there is simply no commitment on the total number of
novels, it may well be one or more. Again, shifting to numerals
greater than or equal to two does not alter this fact. Here, the
alternatives are activated along the dimension of restrictor
predicates (e.g.\ novels, short stories, biographies etc.).
Accordingly, the intonational focus falls over the restrictor. The
English translation would be ``a novel''.

In this connection, I claim that Turkish expression \emph{bir} is not
lexically ambiguous between a numeral and a determiner reading, it has
a single reading that I will formalize below. It results in numeral
(English \emph{one}) and determiner (English \emph{a(n)}) readings
along its interaction with the alternatives provided in the context of
its use.

In summary, I claim that the decision on two binary distinctions,
partitive versus non-partitive and numeral versus determiner are both
left to pragmatics. Now I begin to develop a proposal for the
conventional semantics of \acc-indefinites which will be put to
interaction with context in order to derive the readings we have been
discussing all along.

\subsection{The semantics of indefinite objects}

One way to model ``strength'' in a DRT setting is to adopt
``presuppositionality as anaphoricity'' perspective
(\cttx{sandt92,geurts99}).  According to this view, ``[a] strong
quantifier does not merely presuppose that its domain is non-empty;
rather, the purpose of its presupposition is to \emph{recover} a
suitable domain from the context'' (\cttx[253]{geurts07}). Applying it
to the present case, I claim that:

\ex. The restrictor of an accusative indefinite is anaphoric.


In formal terms, \acc-indefinites gets the following schematic
semantics:

\ex.\label{exsema1}
The semantics of \acc-indefinites (to be revised):\\
\a.\interp{[$n$ N-Acc]} $=$ 
\udrs{\anaph{x}, y}
{\interp{N}(\anaph{x}) \quad n'(y) \quad y \subseteq \anaph{x} }
\b.
\interp{\trtx{bir kızı}{a/one girl.Acc}} $=$ 
\udrs{\anaph{x}, y}
{girls'(\anaph{x}) \quad one'(y) \quad y \subseteq \anaph{x} }
\b.
\interp{\trtx{iki kızı}{two girl.Acc}} $=$ 
\udrs{\anaph{x}, y}
{girls'(\anaph{x}) \quad two'(y) \quad y \subseteq \anaph{x} }

My proposal is in the spirit of \ctnm{enc91}, as it encodes the
semantics of the restrictor and individual introducing parts
separately. An \acc-marked indefinite is inserted to a DRS
with a presuppositional (=anaphoric) restrictor. 

We have two options for the \zero-marked indefinites. The question is
Do we need to separate the restrictor from the referent. I give the
two variants. The only reason for selecting \xref{exsemz11} over
\xref{exsemz12} might be uniformity with \acc-indefinites.

\ex.\label{exsemz1}
The semantics of \zero-indefinites:\\
\a.\label{exsemz11}
\interp{[$n$ N]} $=$ 
\udrs{x, y}
{\interp{N}(x) \quad n'(y) \quad y \subseteq x }
\b.\label{exsemz12}
\interp{[$n$ N]} $=$ 
\udrs{x}
{\interp{N}(x) \quad n'(x)}
%\b.
%\interp{\trtx{bir kız}{a/one girl}} $=$ 
%\udrs{x, y}
%{girls'(x) \quad one'(y) \quad y \subseteq x }
%\b.
%\interp{\trtx{iki kız}{two girl}} $=$ 
%\udrs{x, y}
%{girls'(x) \quad two'(y) \quad y \subseteq x }


Going back to \ctnms{enc91} example, the state of the discourse is as
follows after the indefinite is inserted,

\ex.
\a. Several children entered my room.
\b. I knew two girls.\hfill (Acc-marked)

\ex.
\udrs{s',z,\anaph{x}, y}
{children'(z)\\
girls'(\anaph{x}) \quad two'(y) \quad y \subseteq \anaph{x} \quad know'(s',y)
}

Here the set of girls is presuppositional. As there is no established
set of girls in the context, one needs to accommodate one, in order to
obtain a coherent discourse. I claim that by inference this
accommodated set of girls is most naturally understood to be included
in the set of children introduced in the opening sentence of the
discourse. I do not have a systematic account of why the inference
should yield that result, apart from suggesting that it is the least
costly assumption to make to maintain the coherence of the text.  In
formal terms, \xref{prencaccom} depicts the accommodation of the
antecedent set and the resolution of the presupposition;
\xref{prencfin} depicts the final form of the representation after
eliminating the redundant information:

% TODO: obtain the antecedent by abstraction ala Kamp and Reyle 2010:907.

\ex. 
\a.\label{prencaccom}
\udrs{s',z,\anaph{x},y,v}
{children'(z)\\
girls'(v) \quad v \subseteq z\\
girls'(\anaph{x}) \quad two'(y) \quad y \subseteq \anaph{x} \quad know'(s',y)\\
\anaph{x} = v
}
\b.\label{prencfin}
\udrs{s',z,y,v}
{children'(z)\\
girls'(v) \quad v \subseteq z\\
 two'(y) \quad y \subseteq v \quad know'(s',y)
}


It is by the transitivity of set inclusion that the girls are understood
to belong to the set of children established in the discourse; it does
not follow directly from the semantics of the NP as a definite
superset, contrary to \cite{enc91}. 

Crucially, the present proposal does not generate a partitivity
implication, thanks to the condition `$y\subseteq v$'; if the semantics
were encoding a proper subsethood, then partitivity would follow.

As we saw above, when the restrictors of the indefinite and the antecedent
overlap, we get a partitive implication. Let us observe the present
semantics under such a condition. I simply change the discourse opener
from introducing children to introducing girls.

\ex.
\a. Several girls entered my room.
\b. I knew two girls.\hfill (Acc-marked)


This two sentence discourse receives the following intermediate representation:

\ex.
\udrs{s',z,\anaph{x},y}
{girls'(z)\\
girls'(\anaph{x}) \quad two'(y) \quad y \subseteq \anaph{x} \quad know'(s',y)
}

in which case, accommodation is not called for, since we already have
a suitable antecedent established in the context. The anaphor gets
bound and redundancies are eliminated as follows: 

\ex. 
\a.
\udrs{s',z,\anaph{x},y}
{girls'(z)\\
girls'(\anaph{x}) \quad two'(y) \quad y \subseteq \anaph{x} \quad know'(s',y)\\
\anaph{x} = z
}
\b.
\udrs{s',z,y}
{girls'(z)\\
 two'(y) \quad y \subseteq z \quad know'(s',y)
}


Our semantics does not guarantee partitivity on its own, since the
condition `$y \subseteq z$' does not rule out a model in which the
girls known to the speaker exhaust the girls introduced to the context
via the first sentence. Here, I claim, a Gricean inference is in
action, ruling out this semantically available interpretation: Since
there is a stronger expression that would convey this particular
meaning, namely the plural definite in \xref{expludef} that would
provide this particular meaning; the use of the weaker \emph{two
girls} indicates that the speaker lacks the ground for the use of the
stronger alternative.

\ex.\label{expludef}Kızları tanıyordum.\\
`I knew the girls.'


Now I will consider indefinites carrying \trtx{bir}{`a/one'}. We
discuss the matter over a previous example
(\ref{exromanintro}-\ref{exromancont}), repeated here.

\ex.\label{exromanintrorep}
\a.\label{exromaniromrep}Ali bana bir koli roman bıraktı.\hfill{(identical restrictors)}
\b.\label{exromanikitrep}Ali bana bir koli kitap bıraktı.\hfill{(non-identical restrictors)}

\ex.\label{exromancontrep}
Geçen Pazar bir romanı okudum.\hfill{(\acc-marked continuation)}

We first examine \xref{exromanikitrep} followed by
\xref{exromancontrep}. The state of the discourse after processing
the two sentences is:

\ex. 
\udrs{s',z,\anaph{x},y}
{
books'(z)\\
novels'(\anaph{x}) \quad one'(y) \quad y \subseteq \anaph{x} \quad read'(s',y)
}

A set of novels needs to be accommodated into the discourse, and
this is most easily done by assuming/inferring that the set of books
introduced before contains a subset of novels. After accommodating
such a set, binding can proceed as usual. The end product of these
operations is articulated as follows, where the second DRS is the
result of housekeeping.

\ex.
\a.
\udrs{s',z,\anaph{x},y,v}
{
books'(z)\\
novels'(v) \quad v \subseteq z\\
novels'(\anaph{x}) \quad one'(y) \quad y \subseteq \anaph{x} \quad read'(s',y)\\
\anaph{x} = v
}
\b.
\udrs{s',z,y,v}
{
books'(z)\\
novels'(v) \quad v \subseteq z\\
one'(y) \quad y \subseteq v \quad read'(s',y)
}



Likewise, the discourse comprising of
\xref{exromaniromrep} followed by \xref{exromancontrep}, would yield
the following representation, differing from the previous one by not
involving an ``intermediary'' set of novels.

\ex.
\udrs{s',z,y}
{
novels'(z)\\
one'(y) \quad y \subseteq z \quad read'(s',y)
}

\subsection{Indefinites versus definites}

At this point one might wonder how my formulation for
\emph{bir}-indefinites differs from that of a proper definite, since a
definite description is expected to contribute a ``underlined''
discourse marker whose restrictor is of cardinality one. Here is the
variant at issue:

\ex.\label{exromandef}
\a. Geçen Pazar romanı okudum.


\xref{exromandef} differs from \xref{exromancontrep}. The former, but
not the latter implies a unique identifiability of the novel. Such
identifiability is not afforded by the context and the missing
contextual requirement is more difficult to accommodate in comparison
to the requirement imposed by  \xref{exromancontrep}. The reason is as
follows: A box of books is fairly likely to contain at least one
novel, but it is not likely, without further information, that it
contains a novel uniquely identifiable by the hearer.

The definite in \xref{exromandef} will get the following
interpretation:

\ex.
\udrs{z,\anaph{x}}
{
		books'(z)\\
		novel'(\anaph{x}) \quad one'(\anaph{x}) \quad UNIQ'(\anaph{x}) 
}

Interpretation gets blocked at this point due to the impossibility of
accommodating the presuppostions contributed by the definite. In
present terms, the difference between an \acc-indefinite and a
definite is the unique identifiability constraint contributed by the
latter but not the former.

As \cite{lyons99} observes, definiteness is a multi-faceted notion. In
Turkish we seem to be faced with two varieties of it.  Semantics of
definiteness is usually associated with (subsets of) the following
notions: uniqueness, exhaustivity, identifiability, existence and
familiarity \cttxp{abbott04,heim11}. Some of these dimensions may
apply both at the referent and restrictor level. I will argue that for
an \acc-indefinite, the relevant notions are familiarity and
existence; uniqueness, maximality and identifiability do not play a
role. Furthermore, familiarity and existence involved with
\acc-indefinites apply at the restrictor level only. In this picture
referents are indefinite and the restrictor is definite, akin to
\ctnm{enc91}. For \ctnm{enc91} definiteness involves the existence of
a familiar superset of the restrictor, while in the present account,
what is definite is the restrictor itself, in the sense of familiarity
and existence. In the case of a proper definite (i.e.\ a DP with case
but without the indefinite determiner) we have unique identifiability.

Return to Hawkins 78 p.\ 168 in discussing Enc's formulation; there is a clause:
'There must not be any other objects in the shared set satisfying the
descriptive predicate in addition to those referred to by the definite
description.'




This completes the exposition of the present proposal. In the next
section we will look at how it is applied to a set of constructions.


%\ex.
%\a. Bunların yerine geçen hafta iki roman okudum.
%\b. Bunların yerine geçen hafta iki romanı okudum.

\section{The proposal at work}

\subsection{Hybrid scope}
\label{schybrid}

I now turn back to hybrid scope phenomenon observed in
Section~\ref{scencneg}. 

\ex.\label{exsorudia}
\a. Why did Berta failed?
\b. \label{exsorudiaa}
Belki de komite zor bir soruyu sordu.
\b. \label{exsorudiaz}%
Belki de komite zor bir soru sordu.


The answer \xref{exsorudiaa} differs from the answer \xref{exsorudiaz}
in committing its speaker to the existence of some difficult questions
already in place before the exam. The answer \xref{exsorudiaz} without
case-marking does not carry any such link to the previous discourse.
The \acc-marked version \xref{exsorudiaa} receives the following
initial DRS:\footnote{To keep things tidy, I gloss over the details
associated with \emph{the committee}, taking it simply to be a
discourse referent with an ordinary restrictor, rather than treating
it as a definite description.}

\ex.
\udrs{z}
{
		committee'(z)\\
\udrs[\Diamond]
{y,\anaph{x}}
{
difficult'(\anaph{x})\quad question'(\anaph{x})\quad one'(y)\quad
y\subseteq \anaph{x}\\
asked'(z,y)
}
}

The hearer is forced to accommodate an antecedent for the indefinite's
restrictor and this is preferrably done at the global DRS in the
absence of any conflicting information:\footnote{I skip the steps of
binding and deletion of reducndancies.}

\ex.
\udrs{z,v}
{
		committee'(z)\\
		difficult'(v)\quad question'(v)\\
\udrs[\Diamond]
{y}
{
one'(y)\quad y\subseteq v\\
asked'(z,y)
}
}

Let me return to another previous example, \xref{excond}, repeated
here for convenience:

\ex.\label{excondr}%
Context: John is a referee reviewing an article under discussion.
\ag.\label{excondar}%
	John (kritik) bir hata{\bf-yı} görür-se, makale-yi reddeder.\\
	J. critical a error{\bf-Acc} sees-Cond article-Acc rejects\\
	Rd.\ 1: `If John sees an (critical) error, he'll reject the article.'' (It's
	common ground that there are (critical) errors in the article.)\\
	Rd.\ 2: `An (critical) error is such that if John sees it, he'll reject the
	article.
\bg.\label{excondzr}%
	John (kritik) bir hata görür-se, makale-yi reddeder.\\
	J. critical a error sees-Cond article-Acc rejects\\
	`If John sees an (critical) error, he'll reject the article.'' (no commitment
	to the existence of errors.)



The Reading 1 of \xref{excondar} is represented as:\footnote{I leave
the discussion of Reading 2 to Section~\ref{scwide}.}

\ex.
\a.
\udrs{j',x}{article'(x)\\
\ccon
{\udrs{z,\anaph{y}}{error'(\anaph{y})\quad  z\subseteq \anaph{y} \quad see'(j',z)}}
{\udrs{}{reject'(j',x)}}
{\Rightarrow}
}
\b.
\udrs{j',x,v}{article'(x)\quad error'(v) \quad belong'(v,x)\\
\ccon
{\udrs{z}{z\subseteq v \quad see'(j',z)}}
{\udrs{}{reject'(j',x)}}
{\Rightarrow}
}


Here I modeled the interpretation as accomplished by accommodating a
set of errors belonging to the article. The same utterance is of
course would be appropriate in a context where the existence of errors
in the article were common ground, in which case there would be just
binding without accommodation.

Also, recall that I drew the conclusion in
Section~\ref{sccovertpartitivity} that the link tying \acc-indefinites
to previous discourse need not be one of set membership or an
assignment relation as claimed by \cite{enc91}, but can be quite
flexible. This example testifies positively to this conclusion, as the
accommodated set of errors is inferred to belong to the article, which
is neither a set membership nor an assignment.



In a setting where the presuppositions of an \acc-indefinite is argued
to be satisfied via accommodation in cases where the context is
inadequate in justifying the presupposition, it is expected to observe
non-local accommodation. I will close this section with such an
example. I slightly alter the previous example.  Imagine a context
where an editor is angry with John, who was one of the reviewers of an
article. Someone poses the question why the editor is angry at John.
Here are the candidates for an answer: 

\ex.\label{exinter}
\a.\label{exintera}%
Belki de John kritik bir hatayı görmemiştir.
\b.\label{exinterz}%
Belki de John kritik bir hata görmemiştir.


The \zero-marked variant \xref{exinterz} is fairly odd in this
context: Why should an editor get angry with a referee for not finding
any errors in an article? But the \acc-marked variant \xref{exintera}
is an appropriate answer, it states that John might have failed to
notice a critical error. Crucially, the \acc-marked \xref{exintera} is
appropriate both in a context where it is common ground that there
were critical errors in the article and in a context where it is not.
Our semantics predicts the existence of these possibilities through the
availability of global and intermediate accommodation:

\ex.
\a.
\udrs%
{j'}
{
		\udrs[\Diamond]%
		{}
		{
				\udrs[\neg]%
				{y,\anaph{x}}
				{
						error'(\anaph{x})\quad one'(y)\quad y\subseteq \anaph{x} \\
						see'(j',y)
				}
		}
}\hfill (initial DRS)
\b.
\udrs%
{j',v}
{
		error'(v)\\
		\udrs[\Diamond]%
		{}
		{
				\udrs[\neg]%
				{y}
				{
						one'(y)\quad y\subseteq v \\
						see'(j',y)
				}
		}
}\hfill (global accomodation)
\b.
\udrs%
{j'}
{
		\udrs[\Diamond]%
		{v}
		{
				error'(v)\\
				\udrs[\neg]%
				{y}
				{
						one'(y)\quad y\subseteq v \\
						see'(j',y)
				}
		}
}\hfill (intermediate accomodation)

% TODO: Bir Nazim Hikmet siirine yapilmis ilk beste: Bana Bak Hey Avanak

\subsection{Wide scope}
\label{scwide}
 
Remember that \xref{excondar} repeated here has the reading where the indefinite behaves as if taking wide scope with respect to the conditional.

\ex.\label{excondrr}%
Context: John is a referee reviewing an article under discussion.
\ag.\label{excondarr}%
	John (kritik) bir hata{\bf-yı} görür-se, makale-yi reddeder.\\
	J. critical a error{\bf-Acc} sees-Cond article-Acc rejects\\
	Rd.\ 1: `If John sees an (critical) error, he'll reject the article.'' (It's
	common ground that there are (critical) errors in the article.)\\
	Rd.\ 2: `An (critical) error is such that if John sees it, he'll reject the
	article.




The most straightforward rendering of this reading, given as Rd.\ 2,
would be as follows:

\ex.
\udrs{j',x,z}{article'(x)\\
error'(z)\\ 
\ccon
{\udrs{}{see'(j',z)}}
{\udrs{}{reject'(j',x)}}
{\Rightarrow}
}

which is a standard ``existential quantifier taking scope over
negation'' reading. If this is the right representation for Rd.\ 2,
then Turkish must have an existential quantifier interpretation for an
\acc-indefinite which is forced to take scope over negation. In
Section~\ref{scencneg} we saw that this formulation is not capable of
capturing the facts due to the existence of hybrid scope readings.
Therefore, even if we admit a standard existential reading for
\acc-indefinites, we would still need to have the present formulation
along side with it, thereby admitting an ambiguity for
\acc-indefinites.

Besides the usual hesitations on admitting a potentially dispensable
ambiguity, there is a further problem with having a standard
existential reading for \acc-indefinites. Take the following minimal
pair:

\ex.
\a. Dün gece bir kitap okudum.
\b. Dün gece bir kitabı okudum.


There is simply no operator in these sentences that we can demonstrate
\acc-indefinite taking scope above and the \zero-indefinite taking
scope below. Then why do not we have a synonymy here? 



\ex.
\a.\label{exdrswide1}%
\udrs{j',x}{article'(x)\\
\ccon
{\udrs{z,\anaph{y}}{error'(\anaph{y})\quad  z\subseteq \anaph{y} \quad see'(j',z)}}
{\udrs{}{reject'(j',x)}}
{\Rightarrow}
}\vspace{10pt}
\b.\label{exdrswide2}%
\udrs{j',x,v,z}{article'(x)\quad error'(v) \quad belong'(v,x)\\
z\subseteq v \\ 
\ccon
{\udrs{}{see'(j',z)}}
{\udrs{}{reject'(j',x)}}
{\Rightarrow}
}


It is not possible to obtain \xref{exdrswide2} from \xref{exdrswide1}
only by accommodation and binding, since these mechanisms concern only
the restrictor of the indefinite, not the referent represented by the
discourse marker $z$. Therefore if \xref{exdrswide2} is the right
representation for \xref{exocndarr}, then we would need an independent
mechanism that causes the indefinite to take wide scope with respect
to conditional. In such a setting the derivational steps would be:

\ex.
\a.\label{exdrswidedrv1}%
\udrs{j',x}{article'(x)\\
\ccon
{\udrs{z,\anaph{y}}{error'(\anaph{y})\quad  z\subseteq \anaph{y} \quad see'(j',z)}}
{\udrs{}{reject'(j',x)}}
{\Rightarrow}
}\hfill(Initial representation)\vspace{10pt}
\b.\label{exdrswidedrv2}%
\udrs{j',x,z,\anaph{y}}{article'(x)\\
error'(\anaph{y})\quad  z\subseteq \anaph{y}\\
\ccon
{\udrs{}{see'(j',z)}}
{\udrs{}{reject'(j',x)}}
{\Rightarrow}
}\hfill (Taking wide-scope)\vspace{10pt}
\b.\label{exdrswidedrv2}%
\udrs{j',x,v,z}{article'(x)\quad error'(v) \quad belong'(v,x)\\
z\subseteq v \\ 
\ccon
{\udrs{}{see'(j',z)}}
{\udrs{}{reject'(j',x)}}
{\Rightarrow}
}\hfill (Accomodation and binding)


Have narrowing to a singleton as the model of ``having an individual
in mind''

\subsection{Referentially opaque verbs}
\label{scopaque}

\ctnm[158--9]{dede86} observes that \acc-marked indefinites are bound to
receive transparent interpretations as arguments of referentially
opaque verbs.\footnote{\cttx{dede86} also argues that this behavior
applies only for animate objects, which none of my informants so
far has confirmed.} Here is an example:

\ex.\label{exseek}
\a.\label{exseekz}%
Polis bir çocuk arıyor. Bulamıyor.\hfill(opaque or transparent)\\ 
`The police are seeking a child, but cannot find one/him'.
\b.\label{exseeka}%
Polis bir çocuğu arıyor. Bulamıyor.\hfill(only transparent)\\
`The police are seeking a child, but cannot find *one/him'.


\ctnm{kelepir01} draws attention to the fact that \acc-indefinites
while receiving transparent interpretations do not trigger specificity
effects in the sense of \cttx{enc91}. There is no requirement that the
child sought in \xref{exseeka} be familiar from the previous
discourse.  On the basis of this observation, \ctnm{kelepir01}
rightly, I think, claims that \acc-marking cannot be totally about
linking to discourse.

The present semantics provides an explanation for \ctnms{kelepir01}
observation. The only auxiliary assumption we need is to model
referentially opaque verbs as providing a non-local accommodation site
for the presupposition triggered by the \acc-indefinite. In an
admittedly sketchy formulation, this would look like: 

\ex.
\udrs{x}
{police'(x)\\
		\udrs[<W>]{z}
		{find'(x,z)}
}


\subsection{Lexical matters}

% TODO: Verbs of creation: (iddaname) duzenle/hazirla; non-vreation: temsil
% et, sakla, verbs of destruction, kopyala, (bir turku) yargila. 

\ex. \a.Yarin toplantiya bir moderator getirecegim.
\b. \#Yarin toplantiya bir moderatoru getirecegim.


Ayse elbiselerden iki takim(i) begendi.

Argue over this that the choice of marking affects the argument
structure of the verb -- or the identity of the verb -- you can do
this by introducing small v level boxes. 

% TODO: entity introducing diagnostics from Coppock and Beaver 2015.

bestelemek, bir siir versus bir siiri bestelemek.


Bir Nazim hikmet siirini besteleyen ilk besteci
Bir Nazim hikmet siirini besteleyen sanatci, gazetecilerin sorularini yanitladi.

\section{Some loose ends}
\label{scloose}

\subsection{Intelligent accommodation}

Caution: $V$ should be guaranteed to exhaust the errors in the article
-- presumably by inference.

\subsection{Compositionality}

\intx{bir} is an operator, can be poly-typed or not; it applies to a
bare NP which is a kind term, and carves out a brand-new individual
untied to the context. The same operator applies to an \acc-marked NP
which is in a sense definite. In the absence of \intx{bir} the NP
becomes standard definite due to maximality. In the presence of
\intx{bir} a partitivity is indicated, while the restrictor is still
definite. What do I do with scope then? Either schwarzschild -- for
some reason I do not like that idea, or some other explanation. Other
possibility is this: in the absence of an operator like negation, the
two interpretations collapse. But still we need an explanation for why
the scope is flexible for \acc-marked indefinites. This I can explain
from the other direction. \zero-marked indefinites have an adjacency
condition, they are the exceptional ones, not the \acc-marked ones.
This in turn might be motivated by information structural concerns.    

\subsection{Disjointness condition for Zero-indefinites}

As for the zero marking, I do not propose anything special about it.
It is a standard indefinite analysis and I do not claim a disjointness
semantics. I leave it open that there arise disjointness implicatures,
supported by the existence of a formal device to signal
presuppositionality, it is natural to expect the zero-marked version
to implicate disjointness. 


\subsection{Indefinite to definite}

The ultimate point where context affects the interpretation of an
indefinite is yielding a maximal reading equivalent to that obtained
by a definite description in English. Take the following example.
First a discourse opener:

\ex.\label{exikicocukw}Sınıfın yarısını bana emanet ettiler. 

Then we have the usual minimal pair:

\ex. 
\a.\label{exikicocukz}İki çocuk çalıştırmakta çok zorlandım.
\b.\label{exikicocuka}İki çocuğu çalıştırmakta çok zorlandım.


The \zero-marked \xref{exikicocukz} is again not appropriate without
further assumptions. It roughly states the difficulty of making two
kids work, what is at issue is the number of the kids, it is
implicated that if it were just one kid, the task would be easier.

The \acc-marked \xref{exikicocuka} on the other hand smoothly connects
to the opening sentence \xref{exikicocukw}. The speaker states the
difficulty of making two kids from among the half of the class work.
As the restrictor of the indefinite and the antecedent overlap, we
have a partitive implication. And, again, if the restrictor of the
indefinite were different from that of the antecedent, say \trtx{iki
kızı}{two girl.Acc}, then we would lose this implication. 

Now we consider \ref{exikicocuka} as a continuation to
\ref{exikicocukn}:

\ex.\label{exikicocukn} Elif ve Ayşeyi bana emanet ettiler. 

When posed in this context the same sentence receives a definite
translation. Turkish does not have a definite article that could be
used to disambiguate \xref{exikicocuka}, in the absence of such a
device, it is left to inferential mechanisms to decide on maximality.
If you infer maximality, you get a definite interpretation, otherwise
you get a partitive indefinite interpretation. 



\section{Conclusion}
\label{scconc}

The main arguments and contributions of the paper can be listed as
follows:


The paper argued that the accusative marker in Turkish marks definiteness minus
uniqueness.

Turkish is important as it shows us that various components of the grammar of quantification and reference can have separate morphological reflexes, and thereby provides evidence for a case of compositionality in quantification \cttxp{szabolcsi10}.

% TODO: Why Turkish is important: \cttx{ludlowneale91} \cttx{matthewson99}

% TODO: Start by \cttx[153, Section  9.4]{szabolcsi10}

% TODO: Bir soru(yu) *bile* sormadim.

% TODO: Give a list of interpretative/scopal/IS-related effects in the first section.

% TODO: Acc marking does not necessarily have functional readings (contra KvH).

% TODO: negative verbs require the marker: engelle. durdur, (relate this discussion to verbs of creation).

% TODO: Dün ikişer görevli her hücreyi ziyaret etmiş.

% TODO: strong quans are scope rigid but acc-marked indefs are not.

% TODO: Evidence that Possessives cannot stay VP internal.

% TODO: Acc is not related to scope, it marks an anaphoric dependency of the restrictor.

% TODO: Ali projede iki yabanci(yi) calistiriyor.

% TODO: Ali annesine bir ogrenci*(yi) sikayet etti.

% TODO: Zero do not introduce discourse marker;

Ahmet bana bir cocuk gosterdi; annesini kaybetmis yeni. 

Annesine *Bir cocuk gostermeye gitti.

% TODO: The domain that contains bare objects is sensitive (or closely related) to information structure. Focus domain can expand to left, but it cannot contract to right excluding a bare object.

\begin{comment}

# Ozturk 2005:

main claim: no agree relation from a higher projection in turkish; case is a local phenomenon, assigned in the local projection; case and referentiality are united, therefore no need for a hierarchical configurational phrase structure.
	- Turkish demonstratives, like in Japanese, do not close projections: John'un bu kitabi.
	- She takes 'kirmizi bu kitap' = 'this red book', which is quite controversial.
	- 19-21: why turkish does not have a definite or indefinite determiner.
	- Bare caseless objects are phrasal.
	- p47: they are not arguments; they do not passivize -- they only yield impersonal passives;  they form complex unergative predicates with the verb.
	- p57-8: Doktor hastayi muayene mi etti?/\* Doktor hasta muayene mi etti?
	- p61: Turkish bare NPs are [+pred,-arg], when they are visible for case, they are type-shifted to +arg, otherwise they stay pred and form a complex verb.
	- Case is a type-shifter, it shifts to kinds and definites but not to E.
	- N tane is extractable; 69-70. 
	- WHat unites definties, kinds and spec. indefs. is their being singletons, case can apply to them all.
		- She dismisses Aydemir data with controversial data.
	- She dismisses the possibility of anaphora to bir N's as accommodation.


# Diesing 1992:

* As Diesing has this system for subjects (VP-Internal Subject Hypothesis) it
should apply to objects as well.

* At LF vp int. ext subjects are mapped to quantificational structures.

* Following Heim, restrictive clause -- nuclear scope	partition (has roots in
theme-rheme, topic-comment 	(p.\ 5).

* Existential closure is a last resort operation that prevents unbound variables.

* Strong/weak: Partee 88 few and many; diesing also maintains that there are two
different types of indefinites; presuppositional versus cardinal determiners of
Milsark 1974.

* There is a relation between presuppositionality and obligatoriness of
QR.

* Island constraints are sensitive to LF (p. 13).

* Verbs interacting with the presuppositionality of their objects.

* Individual versus stage level predicates.

> `is ready' might be a counterexample to ind versus stage.

* Subjects of ind-level preds can appear only in the restrictor; subs of
stage-level can appear in both p19.

* The exact position of the VP-internal subject is not important, as long as it is
in the VP domain. p20.

* Stage versus individual is raising versus control.

* Adverbials as VP boundary refs p31.

# Chung and Ladusaw 2004

* Restrict does not saturate an argument, but shifts its lambda to right before the event lambda. 
* A predicate must be fully saturated at the event-level (=when there is nothing left to combine but the inflectional head).
* if any unsaturated argument left at e-level, EC saturates it -- EC may apply sooner.
* negation is interpreted above the event level -- they take this to be the standard view p13.
* ex 29, p13 has an unsaturated argument below event EC -- there is an explanation on the following page that says this does not count as unsaturated -- I don't understand why.
* Specify is CF + App, the choice function is existentially closed non-deterministically (Reinhart, Winter, contra Kratzer and Matthewson)  
* Morphosyntax signals whether to compose with Restrict or Specify.
* he is restricted to subjects.
* some claim tetahi is `the'+`one'.
* Both Maori determiners he and tetahi can be narrow wrt Cond, Neg (see also p.52), Ques, Quan -- for such contexts they are interchangeable p40.
* tetahi can be wide wrt operators but he cannot.
* They return to specificity; external subjects (ind-level, transitive, unergative verbs) must be specific, and only tehati headed NPs can be specific (as they become e type by CF). 
* tehati headed NPs are more prominent in discourse.
* They make lambda shift afer Restrict non-deterministic p109


# Farkas and de Swart 2003

* They distinguish full-fledged narrowest scope (scopal inertia) indefinites from incorporated ones, in contrast to van Geenhoven 1998, who takes the two classes to be identical.
* p104-5 full-fledged indefs cannot scope under negation; this cannot be due to egy (indefinite article) being a PPI, since complex NPs headed by egy can scope under negation. I do not understand the explanation that they base on this fact.  
* TERM: dependent (necessarily co-varying); roofed (require a commanding operator, like NPIs).
* They distinguish between thematic arguments and discourse referents. The former come with predicative categories like V and N, the latter are contributed by determiners.
* In _Mary is a doctor_, _a_ is expletive (de Swart 2001).
* p37, ``The scopal properties of a discourse referent introducing nominal are determined by the item responsible for the introduction of the discourse referent.''

* 51-2 inflection -- like plural in their case -- requires local accommodation, while lexical determiners are more flexible. 

* Plurarlity is an indicator of non-atomicity.

# Lopez 2012:

* syntactic structure limits the availability of modes of combination; its effect is not direct as in diesing92.

\end{comment}

%\begin{udefinition}
%A nominal is weak if its dref gets created at the point it is first evaluated; and there exists no dref that is identical to it in the previous discourse as a variable or as a member of an accessible set. Except accidental coreference.
%\end{udefinition}
%



\setlength{\bibsep}{0pt}
\renewcommand{\bibfont}{\small}
\bibliography{ozge}
\bibliographystyle{apalike}
\end{document}
